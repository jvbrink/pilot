
% Default to the notebook output style

    


% Inherit from the specified cell style.




    
\documentclass[11pt]{article}

    
    
    \usepackage[T1]{fontenc}
    % Nicer default font (+ math font) than Computer Modern for most use cases
    \usepackage{mathpazo}

    % Basic figure setup, for now with no caption control since it's done
    % automatically by Pandoc (which extracts ![](path) syntax from Markdown).
    \usepackage{graphicx}
    % We will generate all images so they have a width \maxwidth. This means
    % that they will get their normal width if they fit onto the page, but
    % are scaled down if they would overflow the margins.
    \makeatletter
    \def\maxwidth{\ifdim\Gin@nat@width>\linewidth\linewidth
    \else\Gin@nat@width\fi}
    \makeatother
    \let\Oldincludegraphics\includegraphics
    % Set max figure width to be 80% of text width, for now hardcoded.
    \renewcommand{\includegraphics}[1]{\Oldincludegraphics[width=.8\maxwidth]{#1}}
    % Ensure that by default, figures have no caption (until we provide a
    % proper Figure object with a Caption API and a way to capture that
    % in the conversion process - todo).
    \usepackage{caption}
    \DeclareCaptionLabelFormat{nolabel}{}
    \captionsetup{labelformat=nolabel}

    \usepackage{adjustbox} % Used to constrain images to a maximum size 
    \usepackage{xcolor} % Allow colors to be defined
    \usepackage{enumerate} % Needed for markdown enumerations to work
    \usepackage{geometry} % Used to adjust the document margins
    \usepackage{amsmath} % Equations
    \usepackage{amssymb} % Equations
    \usepackage{textcomp} % defines textquotesingle
    % Hack from http://tex.stackexchange.com/a/47451/13684:
    \AtBeginDocument{%
        \def\PYZsq{\textquotesingle}% Upright quotes in Pygmentized code
    }
    \usepackage{upquote} % Upright quotes for verbatim code
    \usepackage{eurosym} % defines \euro
    \usepackage[mathletters]{ucs} % Extended unicode (utf-8) support
    \usepackage[utf8x]{inputenc} % Allow utf-8 characters in the tex document
    \usepackage{fancyvrb} % verbatim replacement that allows latex
    \usepackage{grffile} % extends the file name processing of package graphics 
                         % to support a larger range 
    % The hyperref package gives us a pdf with properly built
    % internal navigation ('pdf bookmarks' for the table of contents,
    % internal cross-reference links, web links for URLs, etc.)
    \usepackage{hyperref}
    \usepackage{longtable} % longtable support required by pandoc >1.10
    \usepackage{booktabs}  % table support for pandoc > 1.12.2
    \usepackage[inline]{enumitem} % IRkernel/repr support (it uses the enumerate* environment)
    \usepackage[normalem]{ulem} % ulem is needed to support strikethroughs (\sout)
                                % normalem makes italics be italics, not underlines
    

    
    
    % Colors for the hyperref package
    \definecolor{urlcolor}{rgb}{0,.145,.698}
    \definecolor{linkcolor}{rgb}{.71,0.21,0.01}
    \definecolor{citecolor}{rgb}{.12,.54,.11}

    % ANSI colors
    \definecolor{ansi-black}{HTML}{3E424D}
    \definecolor{ansi-black-intense}{HTML}{282C36}
    \definecolor{ansi-red}{HTML}{E75C58}
    \definecolor{ansi-red-intense}{HTML}{B22B31}
    \definecolor{ansi-green}{HTML}{00A250}
    \definecolor{ansi-green-intense}{HTML}{007427}
    \definecolor{ansi-yellow}{HTML}{DDB62B}
    \definecolor{ansi-yellow-intense}{HTML}{B27D12}
    \definecolor{ansi-blue}{HTML}{208FFB}
    \definecolor{ansi-blue-intense}{HTML}{0065CA}
    \definecolor{ansi-magenta}{HTML}{D160C4}
    \definecolor{ansi-magenta-intense}{HTML}{A03196}
    \definecolor{ansi-cyan}{HTML}{60C6C8}
    \definecolor{ansi-cyan-intense}{HTML}{258F8F}
    \definecolor{ansi-white}{HTML}{C5C1B4}
    \definecolor{ansi-white-intense}{HTML}{A1A6B2}

    % commands and environments needed by pandoc snippets
    % extracted from the output of `pandoc -s`
    \providecommand{\tightlist}{%
      \setlength{\itemsep}{0pt}\setlength{\parskip}{0pt}}
    \DefineVerbatimEnvironment{Highlighting}{Verbatim}{commandchars=\\\{\}}
    % Add ',fontsize=\small' for more characters per line
    \newenvironment{Shaded}{}{}
    \newcommand{\KeywordTok}[1]{\textcolor[rgb]{0.00,0.44,0.13}{\textbf{{#1}}}}
    \newcommand{\DataTypeTok}[1]{\textcolor[rgb]{0.56,0.13,0.00}{{#1}}}
    \newcommand{\DecValTok}[1]{\textcolor[rgb]{0.25,0.63,0.44}{{#1}}}
    \newcommand{\BaseNTok}[1]{\textcolor[rgb]{0.25,0.63,0.44}{{#1}}}
    \newcommand{\FloatTok}[1]{\textcolor[rgb]{0.25,0.63,0.44}{{#1}}}
    \newcommand{\CharTok}[1]{\textcolor[rgb]{0.25,0.44,0.63}{{#1}}}
    \newcommand{\StringTok}[1]{\textcolor[rgb]{0.25,0.44,0.63}{{#1}}}
    \newcommand{\CommentTok}[1]{\textcolor[rgb]{0.38,0.63,0.69}{\textit{{#1}}}}
    \newcommand{\OtherTok}[1]{\textcolor[rgb]{0.00,0.44,0.13}{{#1}}}
    \newcommand{\AlertTok}[1]{\textcolor[rgb]{1.00,0.00,0.00}{\textbf{{#1}}}}
    \newcommand{\FunctionTok}[1]{\textcolor[rgb]{0.02,0.16,0.49}{{#1}}}
    \newcommand{\RegionMarkerTok}[1]{{#1}}
    \newcommand{\ErrorTok}[1]{\textcolor[rgb]{1.00,0.00,0.00}{\textbf{{#1}}}}
    \newcommand{\NormalTok}[1]{{#1}}
    
    % Additional commands for more recent versions of Pandoc
    \newcommand{\ConstantTok}[1]{\textcolor[rgb]{0.53,0.00,0.00}{{#1}}}
    \newcommand{\SpecialCharTok}[1]{\textcolor[rgb]{0.25,0.44,0.63}{{#1}}}
    \newcommand{\VerbatimStringTok}[1]{\textcolor[rgb]{0.25,0.44,0.63}{{#1}}}
    \newcommand{\SpecialStringTok}[1]{\textcolor[rgb]{0.73,0.40,0.53}{{#1}}}
    \newcommand{\ImportTok}[1]{{#1}}
    \newcommand{\DocumentationTok}[1]{\textcolor[rgb]{0.73,0.13,0.13}{\textit{{#1}}}}
    \newcommand{\AnnotationTok}[1]{\textcolor[rgb]{0.38,0.63,0.69}{\textbf{\textit{{#1}}}}}
    \newcommand{\CommentVarTok}[1]{\textcolor[rgb]{0.38,0.63,0.69}{\textbf{\textit{{#1}}}}}
    \newcommand{\VariableTok}[1]{\textcolor[rgb]{0.10,0.09,0.49}{{#1}}}
    \newcommand{\ControlFlowTok}[1]{\textcolor[rgb]{0.00,0.44,0.13}{\textbf{{#1}}}}
    \newcommand{\OperatorTok}[1]{\textcolor[rgb]{0.40,0.40,0.40}{{#1}}}
    \newcommand{\BuiltInTok}[1]{{#1}}
    \newcommand{\ExtensionTok}[1]{{#1}}
    \newcommand{\PreprocessorTok}[1]{\textcolor[rgb]{0.74,0.48,0.00}{{#1}}}
    \newcommand{\AttributeTok}[1]{\textcolor[rgb]{0.49,0.56,0.16}{{#1}}}
    \newcommand{\InformationTok}[1]{\textcolor[rgb]{0.38,0.63,0.69}{\textbf{\textit{{#1}}}}}
    \newcommand{\WarningTok}[1]{\textcolor[rgb]{0.38,0.63,0.69}{\textbf{\textit{{#1}}}}}
    
    
    % Define a nice break command that doesn't care if a line doesn't already
    % exist.
    \def\br{\hspace*{\fill} \\* }
    % Math Jax compatability definitions
    \def\gt{>}
    \def\lt{<}
    % Document parameters
    \title{Opplegg 1 - Tilfeldige Tall og Simuleringer}
    
    
    

    % Pygments definitions
    
\makeatletter
\def\PY@reset{\let\PY@it=\relax \let\PY@bf=\relax%
    \let\PY@ul=\relax \let\PY@tc=\relax%
    \let\PY@bc=\relax \let\PY@ff=\relax}
\def\PY@tok#1{\csname PY@tok@#1\endcsname}
\def\PY@toks#1+{\ifx\relax#1\empty\else%
    \PY@tok{#1}\expandafter\PY@toks\fi}
\def\PY@do#1{\PY@bc{\PY@tc{\PY@ul{%
    \PY@it{\PY@bf{\PY@ff{#1}}}}}}}
\def\PY#1#2{\PY@reset\PY@toks#1+\relax+\PY@do{#2}}

\expandafter\def\csname PY@tok@gd\endcsname{\def\PY@tc##1{\textcolor[rgb]{0.63,0.00,0.00}{##1}}}
\expandafter\def\csname PY@tok@gu\endcsname{\let\PY@bf=\textbf\def\PY@tc##1{\textcolor[rgb]{0.50,0.00,0.50}{##1}}}
\expandafter\def\csname PY@tok@gt\endcsname{\def\PY@tc##1{\textcolor[rgb]{0.00,0.27,0.87}{##1}}}
\expandafter\def\csname PY@tok@gs\endcsname{\let\PY@bf=\textbf}
\expandafter\def\csname PY@tok@gr\endcsname{\def\PY@tc##1{\textcolor[rgb]{1.00,0.00,0.00}{##1}}}
\expandafter\def\csname PY@tok@cm\endcsname{\let\PY@it=\textit\def\PY@tc##1{\textcolor[rgb]{0.25,0.50,0.50}{##1}}}
\expandafter\def\csname PY@tok@vg\endcsname{\def\PY@tc##1{\textcolor[rgb]{0.10,0.09,0.49}{##1}}}
\expandafter\def\csname PY@tok@vi\endcsname{\def\PY@tc##1{\textcolor[rgb]{0.10,0.09,0.49}{##1}}}
\expandafter\def\csname PY@tok@vm\endcsname{\def\PY@tc##1{\textcolor[rgb]{0.10,0.09,0.49}{##1}}}
\expandafter\def\csname PY@tok@mh\endcsname{\def\PY@tc##1{\textcolor[rgb]{0.40,0.40,0.40}{##1}}}
\expandafter\def\csname PY@tok@cs\endcsname{\let\PY@it=\textit\def\PY@tc##1{\textcolor[rgb]{0.25,0.50,0.50}{##1}}}
\expandafter\def\csname PY@tok@ge\endcsname{\let\PY@it=\textit}
\expandafter\def\csname PY@tok@vc\endcsname{\def\PY@tc##1{\textcolor[rgb]{0.10,0.09,0.49}{##1}}}
\expandafter\def\csname PY@tok@il\endcsname{\def\PY@tc##1{\textcolor[rgb]{0.40,0.40,0.40}{##1}}}
\expandafter\def\csname PY@tok@go\endcsname{\def\PY@tc##1{\textcolor[rgb]{0.53,0.53,0.53}{##1}}}
\expandafter\def\csname PY@tok@cp\endcsname{\def\PY@tc##1{\textcolor[rgb]{0.74,0.48,0.00}{##1}}}
\expandafter\def\csname PY@tok@gi\endcsname{\def\PY@tc##1{\textcolor[rgb]{0.00,0.63,0.00}{##1}}}
\expandafter\def\csname PY@tok@gh\endcsname{\let\PY@bf=\textbf\def\PY@tc##1{\textcolor[rgb]{0.00,0.00,0.50}{##1}}}
\expandafter\def\csname PY@tok@ni\endcsname{\let\PY@bf=\textbf\def\PY@tc##1{\textcolor[rgb]{0.60,0.60,0.60}{##1}}}
\expandafter\def\csname PY@tok@nl\endcsname{\def\PY@tc##1{\textcolor[rgb]{0.63,0.63,0.00}{##1}}}
\expandafter\def\csname PY@tok@nn\endcsname{\let\PY@bf=\textbf\def\PY@tc##1{\textcolor[rgb]{0.00,0.00,1.00}{##1}}}
\expandafter\def\csname PY@tok@no\endcsname{\def\PY@tc##1{\textcolor[rgb]{0.53,0.00,0.00}{##1}}}
\expandafter\def\csname PY@tok@na\endcsname{\def\PY@tc##1{\textcolor[rgb]{0.49,0.56,0.16}{##1}}}
\expandafter\def\csname PY@tok@nb\endcsname{\def\PY@tc##1{\textcolor[rgb]{0.00,0.50,0.00}{##1}}}
\expandafter\def\csname PY@tok@nc\endcsname{\let\PY@bf=\textbf\def\PY@tc##1{\textcolor[rgb]{0.00,0.00,1.00}{##1}}}
\expandafter\def\csname PY@tok@nd\endcsname{\def\PY@tc##1{\textcolor[rgb]{0.67,0.13,1.00}{##1}}}
\expandafter\def\csname PY@tok@ne\endcsname{\let\PY@bf=\textbf\def\PY@tc##1{\textcolor[rgb]{0.82,0.25,0.23}{##1}}}
\expandafter\def\csname PY@tok@nf\endcsname{\def\PY@tc##1{\textcolor[rgb]{0.00,0.00,1.00}{##1}}}
\expandafter\def\csname PY@tok@si\endcsname{\let\PY@bf=\textbf\def\PY@tc##1{\textcolor[rgb]{0.73,0.40,0.53}{##1}}}
\expandafter\def\csname PY@tok@s2\endcsname{\def\PY@tc##1{\textcolor[rgb]{0.73,0.13,0.13}{##1}}}
\expandafter\def\csname PY@tok@nt\endcsname{\let\PY@bf=\textbf\def\PY@tc##1{\textcolor[rgb]{0.00,0.50,0.00}{##1}}}
\expandafter\def\csname PY@tok@nv\endcsname{\def\PY@tc##1{\textcolor[rgb]{0.10,0.09,0.49}{##1}}}
\expandafter\def\csname PY@tok@s1\endcsname{\def\PY@tc##1{\textcolor[rgb]{0.73,0.13,0.13}{##1}}}
\expandafter\def\csname PY@tok@dl\endcsname{\def\PY@tc##1{\textcolor[rgb]{0.73,0.13,0.13}{##1}}}
\expandafter\def\csname PY@tok@ch\endcsname{\let\PY@it=\textit\def\PY@tc##1{\textcolor[rgb]{0.25,0.50,0.50}{##1}}}
\expandafter\def\csname PY@tok@m\endcsname{\def\PY@tc##1{\textcolor[rgb]{0.40,0.40,0.40}{##1}}}
\expandafter\def\csname PY@tok@gp\endcsname{\let\PY@bf=\textbf\def\PY@tc##1{\textcolor[rgb]{0.00,0.00,0.50}{##1}}}
\expandafter\def\csname PY@tok@sh\endcsname{\def\PY@tc##1{\textcolor[rgb]{0.73,0.13,0.13}{##1}}}
\expandafter\def\csname PY@tok@ow\endcsname{\let\PY@bf=\textbf\def\PY@tc##1{\textcolor[rgb]{0.67,0.13,1.00}{##1}}}
\expandafter\def\csname PY@tok@sx\endcsname{\def\PY@tc##1{\textcolor[rgb]{0.00,0.50,0.00}{##1}}}
\expandafter\def\csname PY@tok@bp\endcsname{\def\PY@tc##1{\textcolor[rgb]{0.00,0.50,0.00}{##1}}}
\expandafter\def\csname PY@tok@c1\endcsname{\let\PY@it=\textit\def\PY@tc##1{\textcolor[rgb]{0.25,0.50,0.50}{##1}}}
\expandafter\def\csname PY@tok@fm\endcsname{\def\PY@tc##1{\textcolor[rgb]{0.00,0.00,1.00}{##1}}}
\expandafter\def\csname PY@tok@o\endcsname{\def\PY@tc##1{\textcolor[rgb]{0.40,0.40,0.40}{##1}}}
\expandafter\def\csname PY@tok@kc\endcsname{\let\PY@bf=\textbf\def\PY@tc##1{\textcolor[rgb]{0.00,0.50,0.00}{##1}}}
\expandafter\def\csname PY@tok@c\endcsname{\let\PY@it=\textit\def\PY@tc##1{\textcolor[rgb]{0.25,0.50,0.50}{##1}}}
\expandafter\def\csname PY@tok@mf\endcsname{\def\PY@tc##1{\textcolor[rgb]{0.40,0.40,0.40}{##1}}}
\expandafter\def\csname PY@tok@err\endcsname{\def\PY@bc##1{\setlength{\fboxsep}{0pt}\fcolorbox[rgb]{1.00,0.00,0.00}{1,1,1}{\strut ##1}}}
\expandafter\def\csname PY@tok@mb\endcsname{\def\PY@tc##1{\textcolor[rgb]{0.40,0.40,0.40}{##1}}}
\expandafter\def\csname PY@tok@ss\endcsname{\def\PY@tc##1{\textcolor[rgb]{0.10,0.09,0.49}{##1}}}
\expandafter\def\csname PY@tok@sr\endcsname{\def\PY@tc##1{\textcolor[rgb]{0.73,0.40,0.53}{##1}}}
\expandafter\def\csname PY@tok@mo\endcsname{\def\PY@tc##1{\textcolor[rgb]{0.40,0.40,0.40}{##1}}}
\expandafter\def\csname PY@tok@kd\endcsname{\let\PY@bf=\textbf\def\PY@tc##1{\textcolor[rgb]{0.00,0.50,0.00}{##1}}}
\expandafter\def\csname PY@tok@mi\endcsname{\def\PY@tc##1{\textcolor[rgb]{0.40,0.40,0.40}{##1}}}
\expandafter\def\csname PY@tok@kn\endcsname{\let\PY@bf=\textbf\def\PY@tc##1{\textcolor[rgb]{0.00,0.50,0.00}{##1}}}
\expandafter\def\csname PY@tok@cpf\endcsname{\let\PY@it=\textit\def\PY@tc##1{\textcolor[rgb]{0.25,0.50,0.50}{##1}}}
\expandafter\def\csname PY@tok@kr\endcsname{\let\PY@bf=\textbf\def\PY@tc##1{\textcolor[rgb]{0.00,0.50,0.00}{##1}}}
\expandafter\def\csname PY@tok@s\endcsname{\def\PY@tc##1{\textcolor[rgb]{0.73,0.13,0.13}{##1}}}
\expandafter\def\csname PY@tok@kp\endcsname{\def\PY@tc##1{\textcolor[rgb]{0.00,0.50,0.00}{##1}}}
\expandafter\def\csname PY@tok@w\endcsname{\def\PY@tc##1{\textcolor[rgb]{0.73,0.73,0.73}{##1}}}
\expandafter\def\csname PY@tok@kt\endcsname{\def\PY@tc##1{\textcolor[rgb]{0.69,0.00,0.25}{##1}}}
\expandafter\def\csname PY@tok@sc\endcsname{\def\PY@tc##1{\textcolor[rgb]{0.73,0.13,0.13}{##1}}}
\expandafter\def\csname PY@tok@sb\endcsname{\def\PY@tc##1{\textcolor[rgb]{0.73,0.13,0.13}{##1}}}
\expandafter\def\csname PY@tok@sa\endcsname{\def\PY@tc##1{\textcolor[rgb]{0.73,0.13,0.13}{##1}}}
\expandafter\def\csname PY@tok@k\endcsname{\let\PY@bf=\textbf\def\PY@tc##1{\textcolor[rgb]{0.00,0.50,0.00}{##1}}}
\expandafter\def\csname PY@tok@se\endcsname{\let\PY@bf=\textbf\def\PY@tc##1{\textcolor[rgb]{0.73,0.40,0.13}{##1}}}
\expandafter\def\csname PY@tok@sd\endcsname{\let\PY@it=\textit\def\PY@tc##1{\textcolor[rgb]{0.73,0.13,0.13}{##1}}}

\def\PYZbs{\char`\\}
\def\PYZus{\char`\_}
\def\PYZob{\char`\{}
\def\PYZcb{\char`\}}
\def\PYZca{\char`\^}
\def\PYZam{\char`\&}
\def\PYZlt{\char`\<}
\def\PYZgt{\char`\>}
\def\PYZsh{\char`\#}
\def\PYZpc{\char`\%}
\def\PYZdl{\char`\$}
\def\PYZhy{\char`\-}
\def\PYZsq{\char`\'}
\def\PYZdq{\char`\"}
\def\PYZti{\char`\~}
% for compatibility with earlier versions
\def\PYZat{@}
\def\PYZlb{[}
\def\PYZrb{]}
\makeatother


    % Exact colors from NB
    \definecolor{incolor}{rgb}{0.0, 0.0, 0.5}
    \definecolor{outcolor}{rgb}{0.545, 0.0, 0.0}



    
    % Prevent overflowing lines due to hard-to-break entities
    \sloppy 
    % Setup hyperref package
    \hypersetup{
      breaklinks=true,  % so long urls are correctly broken across lines
      colorlinks=true,
      urlcolor=urlcolor,
      linkcolor=linkcolor,
      citecolor=citecolor,
      }
    % Slightly bigger margins than the latex defaults
    
    \geometry{verbose,tmargin=1in,bmargin=1in,lmargin=1in,rmargin=1in}
    
    

    \begin{document}
    
    
    \maketitle
    
    

    
    \section{Opplegg 1 - Tilfeldige Tall og
Simuleringer}\label{opplegg-1---tilfeldige-tall-og-simuleringer}

Tilfeldighet er kjempeviktig i programmering. Tenk deg for eksempel om
du skal lage et dataspill, det hadde være ganske begrenset hva slags
spill du kunne laget om du ikke kan inkludere tilfeldige elementer. Men
det er ikke bare for lek og morro at tilfeldighet er viktig, vi kan også
bruke tilfeldighet for å simulere det som skjer i virkeligheten - på
denne måten kan programmering hjelpe oss å forstå sannsynligheter i
mange forskjellige situasjoner.

\textbf{Plan}

Dette opplegget inneholder informasjon om hvordan vi kan generere
tilfeldige tall i programmene våre, å se hvordan dette kan brukes til å
lage enkle spill og løse noen matematiske problemer. Opplegget avsluttes
med en gjennomgang av Monty Hall problemet, og viser hvordan dette kan
løses med datasimuleringer.

\textbf{Kompetansemål}

\begin{itemize}
\tightlist
\item
  \textbf{Matematikk}

  \begin{enumerate}
  \def\labelenumi{\arabic{enumi}.}
  \tightlist
  \item
    Finne og diskutere sannsyn gjennom eksperimentering, simulering og
    berekning i dagligdagse samanhengar og spel
  \item
    Bruke regneark og graftegner til å utforske tall og variabler og
    presentere resultatene
  \item
    Beskrive utfallsrom og uttrykkje sannsyn som brøk, prosent og
    desimaltal
  \end{enumerate}
\end{itemize}

    

    \subsection{Generere Tilfeldige Tall}\label{generere-tilfeldige-tall}

Det første vi skal se på er hvordan vi kan lage tilfeldige tall. Den
kanskje enkleste kommandoen vi skal bruke heter \texttt{randint()}.
Navnet er litt vanskelig, men \emph{randint} er enn sammensetning av
\emph{random integer}, der \emph{random} betyr tilfeldig på engelsk, og
\emph{integer} betyr heltall, funksjonen gir oss altså et tilfeldig
heltall. Når vi kaller på \texttt{randint()} må vi si noe om hvor mange
muligheter det skal være. La oss se på et eksempel.

\subsubsection{Myntkast}\label{myntkast}

Vi begynner med å se på myntkast, for eksempel for å finne ut hvem som
starter i en fotballkamp. Vi kaster en mynt i været og tar den imot og
ser hvilken side som kom opp. Mynten har to sider, så det er to mulige
utfall, som vi kaller \emph{kron} eller \emph{mynt}. Hvis vi ønsker å
gjennskape myntkast i programmet vårt kaster vi da \texttt{randint(2)},
siden vi har to mulige utfall. 

    \begin{Verbatim}[commandchars=\\\{\}]
{\color{incolor}In [{\color{incolor}2}]:} \PY{k+kn}{from} \PY{n+nn}{pylab} \PY{k}{import} \PY{o}{*}
\end{Verbatim}


    \begin{Verbatim}[commandchars=\\\{\}]
{\color{incolor}In [{\color{incolor}3}]:} \PY{n+nb}{print}\PY{p}{(}\PY{n}{randint}\PY{p}{(}\PY{l+m+mi}{2}\PY{p}{)}\PY{p}{)}
        \PY{n+nb}{print}\PY{p}{(}\PY{n}{randint}\PY{p}{(}\PY{l+m+mi}{2}\PY{p}{)}\PY{p}{)}
        \PY{n+nb}{print}\PY{p}{(}\PY{n}{randint}\PY{p}{(}\PY{l+m+mi}{2}\PY{p}{)}\PY{p}{)}
        \PY{n+nb}{print}\PY{p}{(}\PY{n}{randint}\PY{p}{(}\PY{l+m+mi}{2}\PY{p}{)}\PY{p}{)}
        \PY{n+nb}{print}\PY{p}{(}\PY{n}{randint}\PY{p}{(}\PY{l+m+mi}{2}\PY{p}{)}\PY{p}{)}
\end{Verbatim}


    \begin{Verbatim}[commandchars=\\\{\}]
0
0
1
0
1

    \end{Verbatim}

    Når du kjører koden over ser du at vi på hver linje får skrevet ut enten
\texttt{0} eller \texttt{1}. Hvilken det blir er faktisk helt tilfedig,
og om du kjører koden på nytt vil du få et nytt utfall. I myntkastet
vårt sa vi at utfallene var enten kron eller mynt, men resultatene våre
er enten \texttt{0} eller \texttt{1}, her er det altså viktig at vi før
vi kaster velger hvordan vi skal tolke resultatet. Vi kan for eksempel
før vi "kaster" si at \texttt{0} betyr mynt, og \texttt{1} betyr kron -
det viktige her er at \texttt{randint} gir oss et tilfeldig resultat med
to mulige utfall.

    Men hva om vi ønsker at programmet vårt skal skrive ut til skjerm at det
er mynt eller kron? Det er jo litt slemt mot brukeren av programmet vårt
å la dem bruke tid og krefter på å tolke resultatene. Da kan vi for
eksempel bruke en liste til å gjøre dette som følger:

    \begin{Verbatim}[commandchars=\\\{\}]
{\color{incolor}In [{\color{incolor}4}]:} \PY{n}{utfall} \PY{o}{=} \PY{p}{[}\PY{l+s+s1}{\PYZsq{}}\PY{l+s+s1}{mynt}\PY{l+s+s1}{\PYZsq{}}\PY{p}{,} \PY{l+s+s1}{\PYZsq{}}\PY{l+s+s1}{kron}\PY{l+s+s1}{\PYZsq{}}\PY{p}{]}
        
        \PY{n+nb}{print}\PY{p}{(}\PY{n}{utfall}\PY{p}{[}\PY{n}{randint}\PY{p}{(}\PY{l+m+mi}{2}\PY{p}{)}\PY{p}{]}\PY{p}{)}
        \PY{n+nb}{print}\PY{p}{(}\PY{n}{utfall}\PY{p}{[}\PY{n}{randint}\PY{p}{(}\PY{l+m+mi}{2}\PY{p}{)}\PY{p}{]}\PY{p}{)}
        \PY{n+nb}{print}\PY{p}{(}\PY{n}{utfall}\PY{p}{[}\PY{n}{randint}\PY{p}{(}\PY{l+m+mi}{2}\PY{p}{)}\PY{p}{]}\PY{p}{)}
        \PY{n+nb}{print}\PY{p}{(}\PY{n}{utfall}\PY{p}{[}\PY{n}{randint}\PY{p}{(}\PY{l+m+mi}{2}\PY{p}{)}\PY{p}{]}\PY{p}{)}
        \PY{n+nb}{print}\PY{p}{(}\PY{n}{utfall}\PY{p}{[}\PY{n}{randint}\PY{p}{(}\PY{l+m+mi}{2}\PY{p}{)}\PY{p}{]}\PY{p}{)}
\end{Verbatim}


    \begin{Verbatim}[commandchars=\\\{\}]
kron
kron
mynt
mynt
kron

    \end{Verbatim}

    Her lager vi først en liste med utfallene vi har, så bruker vi
\texttt{randint} til å trekke et av dem tilfeldig. Her er det lurt å
tenke steg for steg hva som skjer. Først trekker \texttt{randint(2)}
enten \texttt{0} eller \texttt{1}, og deretter brukes dette tallet som
\emph{indeks} for å velge et element i lista. Husk at
\texttt{utfall{[}0{]}} betyr første element i lista, så her blir det til
"mynt" og \texttt{utfall{[}1{]}} betyr andre element i lista, som da
blir "kron".

    \textbf{Eksempeloppgave:} Vi ønsker å ha en funksjon som simulerer et
myntkast. Kall funksjonen \texttt{myntkast()}, den skal ikke ta noen
input, og gi enten \texttt{kron} eller \texttt{mynt} tilbake.

\textbf{Fasit:}

    \begin{Verbatim}[commandchars=\\\{\}]
{\color{incolor}In [{\color{incolor}6}]:} \PY{k}{def} \PY{n+nf}{myntkast}\PY{p}{(}\PY{p}{)}\PY{p}{:}
            \PY{n}{utfall} \PY{o}{=} \PY{p}{[}\PY{l+s+s1}{\PYZsq{}}\PY{l+s+s1}{mynt}\PY{l+s+s1}{\PYZsq{}}\PY{p}{,} \PY{l+s+s1}{\PYZsq{}}\PY{l+s+s1}{kron}\PY{l+s+s1}{\PYZsq{}}\PY{p}{]}
            \PY{k}{return} \PY{n}{utfall}\PY{p}{[}\PY{n}{randint}\PY{p}{(}\PY{l+m+mi}{2}\PY{p}{)}\PY{p}{]}
\end{Verbatim}


    Med denne funksjonen er det nå veldig lett å kaste mynt mange ganger. La
oss si vi ønsker å kaste mynt 10 ganger, da bare kaller vi på funksjonen
ti ganger på rad. Når vi ønsker å gjenta noe mange ganger er det smart å
bruke en løkke, så vi slipper å skrive \texttt{myntkast()} gang på gang
på gang.

    \begin{Verbatim}[commandchars=\\\{\}]
{\color{incolor}In [{\color{incolor}7}]:} \PY{c+c1}{\PYZsh{} Vi vil kaste 10 ganger på rad, så vi bruker en løkke}
        \PY{k}{for} \PY{n}{kast} \PY{o+ow}{in} \PY{n+nb}{range}\PY{p}{(}\PY{l+m+mi}{10}\PY{p}{)}\PY{p}{:}
            \PY{n+nb}{print}\PY{p}{(}\PY{n}{myntkast}\PY{p}{(}\PY{p}{)}\PY{p}{)}
\end{Verbatim}


    \begin{Verbatim}[commandchars=\\\{\}]
kron
mynt
mynt
kron
mynt
mynt
mynt
kron
kron
kron

    \end{Verbatim}

    Vi ser at funksjonen og løkken gjør det veldig lett for oss å kaste mynt
mange ganger på rad. Å kaste en mynt 10 ganger på rad er jo ikke så
vanskelig å gjøre for hånd. Men si vi ønsker å kaste 100 ganger, eller
1000. I programmet vårt er det bare å legge på en 0 eller to og kjøre på
nytt, men det tar fort veldig lang tid å gjøre for hånd!

    \subsubsection{Simulere ett veddemål}\label{simulere-ett-veddemuxe5l}

La oss si vi flipper en mynt 6 ganger på rad. Hvor mange kron, og hvor
mange mynt forventer vi å få? Vel, det er like sannsynlig å få begge
deler, så vi kan kanskje gjette på at det å få 3 av hver er mest
sannsynlig. Men hvor sannsynlig er det?

For eksempel kan jeg foreslå følgende veddemål: Vi flipper 6 mynter, om
det blir 3 kron og 3 mynt får du 10 kroner av meg, men om det blir noe
annet, for eksempel 2 kron og 4 mynt, så må du gi meg 10 kroner. Er
dette et bra veddemål for deg? Dette kan vi sjekke med programmering.

    Vi har sett vi kan bruke en løkke til å gjøre mange myntkast, men det å
telle over resultatene skrevet ut over mange linjer er fort mye jobb, og
det er lett å telle feil. La oss derfor sørge for at programmet vårt
holder tellingen automatisk.

Vi lager derfor to tellevariabler som kan holde tellingen på hvor mange
ganger vi har fått kron og mynt. Hver gang vi kaster en mynt kan vi
bruke en \texttt{if}-test for å sjekke resultatet og øke riktig
tellevariabel. Til slutt skriver vi ut tellevariablene våre for å se på
sluttresultatet.

    \begin{Verbatim}[commandchars=\\\{\}]
{\color{incolor}In [{\color{incolor}27}]:} \PY{c+c1}{\PYZsh{} Variabler for å holde tellingen}
         \PY{n}{mynt} \PY{o}{=} \PY{l+m+mi}{0}
         \PY{n}{kron} \PY{o}{=} \PY{l+m+mi}{0}
         
         \PY{c+c1}{\PYZsh{} Vi kaster 6 mynter}
         \PY{k}{for} \PY{n}{kast} \PY{o+ow}{in} \PY{n+nb}{range}\PY{p}{(}\PY{l+m+mi}{6}\PY{p}{)}\PY{p}{:}
             \PY{n}{resultat} \PY{o}{=} \PY{n}{myntkast}\PY{p}{(}\PY{p}{)}
             
             \PY{c+c1}{\PYZsh{} Bruker en if\PYZhy{}else test for å sjekke resultatet}
             \PY{k}{if} \PY{n}{resultat} \PY{o}{==} \PY{l+s+s2}{\PYZdq{}}\PY{l+s+s2}{mynt}\PY{l+s+s2}{\PYZdq{}}\PY{p}{:}
                 \PY{n}{mynt} \PY{o}{+}\PY{o}{=} \PY{l+m+mi}{1}
             \PY{k}{else}\PY{p}{:}
                 \PY{n}{kron} \PY{o}{+}\PY{o}{=} \PY{l+m+mi}{1}
         
         \PY{c+c1}{\PYZsh{} Skriv ut resultatene}
         \PY{n+nb}{print}\PY{p}{(}\PY{l+s+s2}{\PYZdq{}}\PY{l+s+s2}{Mynt:}\PY{l+s+s2}{\PYZdq{}}\PY{p}{,} \PY{n}{mynt}\PY{p}{)}
         \PY{n+nb}{print}\PY{p}{(}\PY{l+s+s2}{\PYZdq{}}\PY{l+s+s2}{Kron:}\PY{l+s+s2}{\PYZdq{}}\PY{p}{,} \PY{n}{kron}\PY{p}{)}
\end{Verbatim}


    \begin{Verbatim}[commandchars=\\\{\}]
Mynt: 3
Kron: 3

    \end{Verbatim}

    Det vi gjør når vi kjører et program av denne typen er å gjennomføre det
vi kaller en \emph{datasimulering}, det betyr at vi gjenskaper eller
imiterer en prosess i et dataprogram.

Datasimuleringen vi har gjort har gjennomført veddemålet vi foreslo
over, fordi vi har kastet 6 mynter og skriver ut antallet kron og mynt.
Du kan nå kjøre programmet en del ganger for å se hvor lett det er å få
nøyaktig 3 kron og 3 mynt, og hvor ofte du bommer.

    \textbf{Eksempeloppgave:} Før vi går videre kan det være lurt å lage en
funksjon som kaster en mynt et gitt antall ganger og gir oss fordelingen
av kron og mynt - lag denne funksjonen ved å fylle inn i skjellettkoden.
Kall den \texttt{mange\_myntkast(n)}, der \texttt{n} er antall kast vi
skal ha. Funksjonen skal returnere antall mynt og kron man har fått.

\textbf{Skjelettkode:}

\begin{verbatim}
def mange_myntkast(n):
    ...
    ...
    ...
    
    return mynt, kron
\end{verbatim}

\textbf{Fasit:}

    \begin{Verbatim}[commandchars=\\\{\}]
{\color{incolor}In [{\color{incolor}30}]:} \PY{k}{def} \PY{n+nf}{mange\PYZus{}myntkast}\PY{p}{(}\PY{n}{n}\PY{p}{)}\PY{p}{:}
             \PY{n}{mynt} \PY{o}{=} \PY{l+m+mi}{0}
             \PY{n}{kron} \PY{o}{=} \PY{l+m+mi}{0}
             \PY{k}{for} \PY{n}{kast} \PY{o+ow}{in} \PY{n+nb}{range}\PY{p}{(}\PY{n}{n}\PY{p}{)}\PY{p}{:}
                 \PY{n}{resultat} \PY{o}{=} \PY{n}{myntkast}\PY{p}{(}\PY{p}{)}
                 \PY{k}{if} \PY{n}{resultat} \PY{o}{==} \PY{l+s+s1}{\PYZsq{}}\PY{l+s+s1}{mynt}\PY{l+s+s1}{\PYZsq{}}\PY{p}{:}
                     \PY{n}{mynt} \PY{o}{+}\PY{o}{=} \PY{l+m+mi}{1}
                 \PY{k}{else}\PY{p}{:}
                     \PY{n}{kron} \PY{o}{+}\PY{o}{=} \PY{l+m+mi}{1}
             \PY{k}{return} \PY{n}{mynt}\PY{p}{,} \PY{n}{kron}
\end{Verbatim}


    \begin{Verbatim}[commandchars=\\\{\}]
{\color{incolor}In [{\color{incolor}31}]:} \PY{n+nb}{print}\PY{p}{(}\PY{n}{mange\PYZus{}myntkast}\PY{p}{(}\PY{l+m+mi}{6}\PY{p}{)}\PY{p}{)}
         \PY{n+nb}{print}\PY{p}{(}\PY{n}{mange\PYZus{}myntkast}\PY{p}{(}\PY{l+m+mi}{6}\PY{p}{)}\PY{p}{)}
         \PY{n+nb}{print}\PY{p}{(}\PY{n}{mange\PYZus{}myntkast}\PY{p}{(}\PY{l+m+mi}{6}\PY{p}{)}\PY{p}{)}
\end{Verbatim}


    \begin{Verbatim}[commandchars=\\\{\}]
(5, 1)
(1, 5)
(2, 4)

    \end{Verbatim}

    \subsubsection{Finne Sannsynlighet}\label{finne-sannsynlighet}

Vi har nå skrevet en funksjon som flipper mynter og gir oss fordelingen
tilbake, og ved å kjøre funksjonen mange ganger kan vi få en viss
følelse av hvor sannsynlig det er å få akkurat 3 mynt og 3 kron. Men la
oss gå et steg lenger å finne den faktiske sannsynligheten, altså
akkurat hvor mange prosent sannsynlig det er å få 3 kron og 3 mynt.

    Sannsynlighetsformelen er som følger
\[{\rm sannsynlighet} = \frac{\text{antall gunstige utfall}}{\text{antall mulige utfall}}.\]

For en datasimulering, som vi bruker, kan vi erstatte denne formelen med
\[{\rm sannsynlighet} = \frac{\text{antall simuleringer med gunstig utfall}}{\text{antall simuleringer totalt}}.\]
Denne formelen fungerer bra om vi gjør nok simuleringer, men det kommer
vi litt tilbake til.

La oss prøve å gjøre 1000 simuleringer, og telle hvor mange hvor vi får
nøyaktig 3 krong og 3 mynt:

    \begin{Verbatim}[commandchars=\\\{\}]
{\color{incolor}In [{\color{incolor}43}]:} \PY{c+c1}{\PYZsh{} Tellevariabel}
         \PY{n}{gunstige\PYZus{}simuleringer} \PY{o}{=} \PY{l+m+mi}{0}
         \PY{n}{antall\PYZus{}simuleringer} \PY{o}{=} \PY{l+m+mi}{1000}
         
         \PY{c+c1}{\PYZsh{} Gjør 1000 simuleringer}
         \PY{k}{for} \PY{n}{simulering} \PY{o+ow}{in} \PY{n+nb}{range}\PY{p}{(}\PY{n}{antall\PYZus{}simuleringer}\PY{p}{)}\PY{p}{:}
             \PY{n}{mynt}\PY{p}{,} \PY{n}{kron} \PY{o}{=} \PY{n}{mange\PYZus{}myntkast}\PY{p}{(}\PY{l+m+mi}{6}\PY{p}{)}
             \PY{k}{if} \PY{n}{mynt} \PY{o}{==} \PY{l+m+mi}{3} \PY{o+ow}{and} \PY{n}{kron} \PY{o}{==} \PY{l+m+mi}{3}\PY{p}{:}
                 \PY{n}{gunstige\PYZus{}simuleringer} \PY{o}{+}\PY{o}{=} \PY{l+m+mi}{1}
         
         
         \PY{n}{sannsynlighet} \PY{o}{=} \PY{n}{gunstige\PYZus{}simuleringer}\PY{o}{/}\PY{n}{antall\PYZus{}simuleringer}
                 
         \PY{n+nb}{print}\PY{p}{(}\PY{l+s+s2}{\PYZdq{}}\PY{l+s+s2}{Antall gunstige simuleringer:}\PY{l+s+s2}{\PYZdq{}}\PY{p}{,} \PY{n}{gunstige\PYZus{}simuleringer}\PY{p}{)}
         \PY{n+nb}{print}\PY{p}{(}\PY{l+s+s2}{\PYZdq{}}\PY{l+s+s2}{Sannsynlighet:}\PY{l+s+s2}{\PYZdq{}}\PY{p}{,} \PY{n}{sannsynlighet}\PY{p}{)}
\end{Verbatim}


    \begin{Verbatim}[commandchars=\\\{\}]
Antall gunstige simuleringer: 302
Sannsynlighet: 0.302

    \end{Verbatim}

    Her har vi først valgt å skrive ut hvor mange av simuleringene som var
gunstige. Gunstige her betyr at vi får nøyaktig 3 kron og 3 mynt, for da
vinner man jo veddemålet. Deretter skriver vi ut sannsynligheten ved å
dele på det totale antallet simuleringer.

    Sannsynligheten her gies som en et tall mellom 0 og 1, og hvis vi kjører
programmet på nytt vil det endre seg litt - dette er fordi vi gjør en
tilfeldig simulering, så eksperimentet vårt gjøres helt på nytt hver
gang, og får litt forskjellige utfall.

For å skrive ut sannsynligheten som en prosent, så kan vi bruke
print-formattering, som følger:

    \begin{Verbatim}[commandchars=\\\{\}]
{\color{incolor}In [{\color{incolor}45}]:} \PY{n+nb}{print}\PY{p}{(}\PY{l+s+s2}{\PYZdq{}}\PY{l+s+si}{\PYZob{}:.2\PYZpc{}\PYZcb{}}\PY{l+s+s2}{\PYZdq{}}\PY{o}{.}\PY{n}{format}\PY{p}{(}\PY{n}{gunstige\PYZus{}simuleringer}\PY{o}{/}\PY{n}{antall\PYZus{}simuleringer}\PY{p}{)}\PY{p}{)}
\end{Verbatim}


    \begin{Verbatim}[commandchars=\\\{\}]
30.20\%

    \end{Verbatim}

    Her betyr ":.2\%" at vi ønsker å skrive ut tallet som en prosent med 2
desimaler. Python ganger da tallet vårt med "100 \%" og runder av til to
desimaler for oss. Vi kunne også gjort dette manuelt, om vi syns det er
lettere

    \begin{Verbatim}[commandchars=\\\{\}]
{\color{incolor}In [{\color{incolor}46}]:} \PY{n+nb}{print}\PY{p}{(}\PY{l+m+mi}{100}\PY{o}{*}\PY{n}{gunstige\PYZus{}simuleringer}\PY{o}{/}\PY{n}{antall\PYZus{}simuleringer}\PY{p}{,} \PY{l+s+s2}{\PYZdq{}}\PY{l+s+s2}{\PYZpc{}}\PY{l+s+s2}{\PYZdq{}}\PY{p}{)}
\end{Verbatim}


    \begin{Verbatim}[commandchars=\\\{\}]
30.2 \%

    \end{Verbatim}

    Vi ser altså at det er omtrent 30\% sannsynlighet for at vi får akkurat
3 kron og 3 mynt. Det betyr at vi ikke bør si ja til veddemålet, fordi
man vil tape mer penger enn man tjener om man spiller mange ganger.

    \subsection{Rulle terninger}\label{rulle-terninger}

Vi har nå sett på myntkast, la oss prøve å rulle en terning istedet. Det
finnes mange forskjellige typer terninger i verden, men la oss starte
med en helt vanlig 6-sidet terning, som av og til kalles for en "d6" i
spillverden, "d" for "dice" som er engelsk for terning, og 6 fordi det
er 6 mulige utfall.

En 6-sidet terning har 6 mulige utfall, så vi kan prøve å bruke
\texttt{randint} igjen, da kaller vi på \texttt{randint(6)} istedetfor
\texttt{randint(2)}.

    \begin{Verbatim}[commandchars=\\\{\}]
{\color{incolor}In [{\color{incolor}10}]:} \PY{n+nb}{print}\PY{p}{(}\PY{n}{randint}\PY{p}{(}\PY{l+m+mi}{6}\PY{p}{)}\PY{p}{)}
         \PY{n+nb}{print}\PY{p}{(}\PY{n}{randint}\PY{p}{(}\PY{l+m+mi}{6}\PY{p}{)}\PY{p}{)}
         \PY{n+nb}{print}\PY{p}{(}\PY{n}{randint}\PY{p}{(}\PY{l+m+mi}{6}\PY{p}{)}\PY{p}{)}
         \PY{n+nb}{print}\PY{p}{(}\PY{n}{randint}\PY{p}{(}\PY{l+m+mi}{6}\PY{p}{)}\PY{p}{)}
         \PY{n+nb}{print}\PY{p}{(}\PY{n}{randint}\PY{p}{(}\PY{l+m+mi}{6}\PY{p}{)}\PY{p}{)}
\end{Verbatim}


    \begin{Verbatim}[commandchars=\\\{\}]
0
4
1
1
3

    \end{Verbatim}

    Som før, om du kjører på nytt vil du få nye utfall. Om du prøver et par
ganger ser du fort at vi får 6 forskjellige utfall, akkurat slik vi vil.
Derimot får vi resultater fra 0 til 5, mens verdien på en vanlig terning
er 1 til 6. Akkurat som for myntkast må vi lage en tolkning av
resultatet vårt.

Vi har to måter vi kan få koden til å skrive ut resultater fra 1 til 6.
Den første er rett og slett å bare legge til 1 til resultatet \[
0+1 \to 1 \\
1+1 \to 2 \\
2+1 \to 3 \\
3+1 \to 4 \\
4+1 \to 5 \\
5+1 \to 6 
\]

    \begin{Verbatim}[commandchars=\\\{\}]
{\color{incolor}In [{\color{incolor}11}]:} \PY{n+nb}{print}\PY{p}{(}\PY{n}{randint}\PY{p}{(}\PY{l+m+mi}{6}\PY{p}{)}\PY{o}{+}\PY{l+m+mi}{1}\PY{p}{)}
         \PY{n+nb}{print}\PY{p}{(}\PY{n}{randint}\PY{p}{(}\PY{l+m+mi}{6}\PY{p}{)}\PY{o}{+}\PY{l+m+mi}{1}\PY{p}{)}
         \PY{n+nb}{print}\PY{p}{(}\PY{n}{randint}\PY{p}{(}\PY{l+m+mi}{6}\PY{p}{)}\PY{o}{+}\PY{l+m+mi}{1}\PY{p}{)}
         \PY{n+nb}{print}\PY{p}{(}\PY{n}{randint}\PY{p}{(}\PY{l+m+mi}{6}\PY{p}{)}\PY{o}{+}\PY{l+m+mi}{1}\PY{p}{)}
         \PY{n+nb}{print}\PY{p}{(}\PY{n}{randint}\PY{p}{(}\PY{l+m+mi}{6}\PY{p}{)}\PY{o}{+}\PY{l+m+mi}{1}\PY{p}{)}
\end{Verbatim}


    \begin{Verbatim}[commandchars=\\\{\}]
2
3
2
2
6

    \end{Verbatim}

    En annen løsning er at vi kan gi to tall til
\texttt{randint}-funksjonen. Vi har sett at om vi bare gir ett tall til
randint gir den tall fra 0 opp til, men ikke, med, tallet vi gir. Så
\texttt{randint(6)} gir tall fra 0 til og med 5. Om vi gir to tall
derimot, tolkes dette litt annerledes. Om vi skriver
\texttt{randint(start,\ stopp)} får vi et tall \textbf{fra og med}
\texttt{start}, til (\textbf{men ikke med}) \texttt{stopp}. Så om vi
skriver \texttt{randint(1,\ 7)} får vi tall fra og med 1 til og med 6,
akkurat slik vi vil.

    \begin{Verbatim}[commandchars=\\\{\}]
{\color{incolor}In [{\color{incolor}12}]:} \PY{n+nb}{print}\PY{p}{(}\PY{n}{randint}\PY{p}{(}\PY{l+m+mi}{1}\PY{p}{,} \PY{l+m+mi}{7}\PY{p}{)}\PY{p}{)}
         \PY{n+nb}{print}\PY{p}{(}\PY{n}{randint}\PY{p}{(}\PY{l+m+mi}{1}\PY{p}{,} \PY{l+m+mi}{7}\PY{p}{)}\PY{p}{)}
         \PY{n+nb}{print}\PY{p}{(}\PY{n}{randint}\PY{p}{(}\PY{l+m+mi}{1}\PY{p}{,} \PY{l+m+mi}{7}\PY{p}{)}\PY{p}{)}
         \PY{n+nb}{print}\PY{p}{(}\PY{n}{randint}\PY{p}{(}\PY{l+m+mi}{1}\PY{p}{,} \PY{l+m+mi}{7}\PY{p}{)}\PY{p}{)}
         \PY{n+nb}{print}\PY{p}{(}\PY{n}{randint}\PY{p}{(}\PY{l+m+mi}{1}\PY{p}{,} \PY{l+m+mi}{7}\PY{p}{)}\PY{p}{)}
\end{Verbatim}


    \begin{Verbatim}[commandchars=\\\{\}]
4
4
2
6
5

    \end{Verbatim}

    Om du foretrekker å skrive \texttt{randint(6)\ +\ 1} eller
\texttt{randint(1,\ 7)} er helt opp til deg, de to gir helt likt svar.
Dette er en vanlig gjenganger i programmering, det finnes som oftest
mange forskjellige måter å gjøre ting på, som gir samme resultat.

    \textbf{Eksempeloppgave:} Vi kan lage en funksjon som ruller terning for
oss. Kall den \texttt{kast\_1d6()} fordi den kaster nøyaktig én 6-sidet
terning. Funksjonen skal returnere et tall fra og med 1 til og med 6.

\textbf{Fasit:}

    \begin{Verbatim}[commandchars=\\\{\}]
{\color{incolor}In [{\color{incolor}48}]:} \PY{k}{def} \PY{n+nf}{kast\PYZus{}1d6}\PY{p}{(}\PY{p}{)}\PY{p}{:}
             \PY{k}{return} \PY{n}{randint}\PY{p}{(}\PY{l+m+mi}{6}\PY{p}{)} \PY{o}{+} \PY{l+m+mi}{1}
\end{Verbatim}


    \begin{Verbatim}[commandchars=\\\{\}]
{\color{incolor}In [{\color{incolor}49}]:} \PY{n+nb}{print}\PY{p}{(}\PY{n}{kast\PYZus{}1d6}\PY{p}{(}\PY{p}{)}\PY{p}{)}
         \PY{n+nb}{print}\PY{p}{(}\PY{n}{kast\PYZus{}1d6}\PY{p}{(}\PY{p}{)}\PY{p}{)}
         \PY{n+nb}{print}\PY{p}{(}\PY{n}{kast\PYZus{}1d6}\PY{p}{(}\PY{p}{)}\PY{p}{)}
\end{Verbatim}


    \begin{Verbatim}[commandchars=\\\{\}]
2
1
6

    \end{Verbatim}

    \textbf{Eksempeloppgave:} Nå kan du lage funksjoner som ruller 2 eller 3
terninger og returnerer summen. Kall disse \texttt{kast\_2d6()} og
\texttt{kast\_3d6()}.

\textbf{Fasit:}

    \begin{Verbatim}[commandchars=\\\{\}]
{\color{incolor}In [{\color{incolor}50}]:} \PY{k}{def} \PY{n+nf}{kast\PYZus{}2d6}\PY{p}{(}\PY{p}{)}\PY{p}{:}
             \PY{k}{return} \PY{n}{kast\PYZus{}1d6}\PY{p}{(}\PY{p}{)} \PY{o}{+} \PY{n}{kast\PYZus{}1d6}\PY{p}{(}\PY{p}{)}
         
         \PY{k}{def} \PY{n+nf}{kast\PYZus{}3d6}\PY{p}{(}\PY{p}{)}\PY{p}{:}
             \PY{k}{return} \PY{n}{kast\PYZus{}1d6}\PY{p}{(}\PY{p}{)} \PY{o}{+} \PY{n}{kast\PYZus{}1d6}\PY{p}{(}\PY{p}{)} \PY{o}{+} \PY{n}{kast\PYZus{}1d6}\PY{p}{(}\PY{p}{)}
\end{Verbatim}


    \begin{Verbatim}[commandchars=\\\{\}]
{\color{incolor}In [{\color{incolor}51}]:} \PY{n+nb}{print}\PY{p}{(}\PY{n}{kast\PYZus{}2d6}\PY{p}{(}\PY{p}{)}\PY{p}{)}
         \PY{n+nb}{print}\PY{p}{(}\PY{n}{kast\PYZus{}2d6}\PY{p}{(}\PY{p}{)}\PY{p}{)}
         \PY{n+nb}{print}\PY{p}{(}\PY{n}{kast\PYZus{}2d6}\PY{p}{(}\PY{p}{)}\PY{p}{)}
\end{Verbatim}


    \begin{Verbatim}[commandchars=\\\{\}]
3
3
6

    \end{Verbatim}

    \begin{Verbatim}[commandchars=\\\{\}]
{\color{incolor}In [{\color{incolor}52}]:} \PY{n+nb}{print}\PY{p}{(}\PY{n}{kast\PYZus{}3d6}\PY{p}{(}\PY{p}{)}\PY{p}{)}
         \PY{n+nb}{print}\PY{p}{(}\PY{n}{kast\PYZus{}3d6}\PY{p}{(}\PY{p}{)}\PY{p}{)}
         \PY{n+nb}{print}\PY{p}{(}\PY{n}{kast\PYZus{}3d6}\PY{p}{(}\PY{p}{)}\PY{p}{)}
\end{Verbatim}


    \begin{Verbatim}[commandchars=\\\{\}]
8
9
14

    \end{Verbatim}

    \textbf{Eksempeloppgave:} Vi kan også lage en funksjon som kan rulle
veldig mange terninger. Kall denne \texttt{kast\_Nd6(N)}, der \texttt{N}
er antall terninger som skal rulles. Funksjonen skal returnere summen av
alle kastene. \emph{Hint:} Bruk en løkke og en tellevariabel.

    \begin{Verbatim}[commandchars=\\\{\}]
{\color{incolor}In [{\color{incolor}53}]:} \PY{k}{def} \PY{n+nf}{kast\PYZus{}Nd6}\PY{p}{(}\PY{n}{N}\PY{p}{)}\PY{p}{:}
             \PY{n}{total} \PY{o}{=} \PY{l+m+mi}{0}
             \PY{k}{for} \PY{n}{kast} \PY{o+ow}{in} \PY{n+nb}{range}\PY{p}{(}\PY{n}{N}\PY{p}{)}\PY{p}{:}
                 \PY{n}{total} \PY{o}{+}\PY{o}{=} \PY{n}{kast\PYZus{}1d6}\PY{p}{(}\PY{p}{)}
             \PY{k}{return} \PY{n}{total}
\end{Verbatim}


    \begin{Verbatim}[commandchars=\\\{\}]
{\color{incolor}In [{\color{incolor}55}]:} \PY{n}{kast\PYZus{}Nd6}\PY{p}{(}\PY{l+m+mi}{10}\PY{p}{)}
\end{Verbatim}


\begin{Verbatim}[commandchars=\\\{\}]
{\color{outcolor}Out[{\color{outcolor}55}]:} 33
\end{Verbatim}
            
    La oss si at noen foreslår et spill. Du betaler først 10 kroner for å
spille, og så ruller 2 terninger. Hvis summen av terningen er 8 eller
lavere vinner du ingenting, du bare mister penger. Hvis du ruller 9, 10
eller 11 vinner du 20 kroner (så en fortjeneste på 10 kroner), og hvis
du ruller 12 vinner du 100 kroner (en fortjeneste på 90 kroner). Er
dette spillet lønnsomt?

La oss lage spillet som en simulering og teste. Vi lager først en
funksjon som spiller spillet en gang, vi lar funksjonen returnere
mengden penger vi taper eller vinner. Hvis vi ruller 8 eller mindre,
taper vi 10. Ruller vi over 8, men under 12, så vinner vi 10, og om vi
ruller 12 får vi 90 kroner fortjeneste (batler 10 kronger, men vinner
100 betyr en fortjenste på 100-10 = 90).

    \begin{Verbatim}[commandchars=\\\{\}]
{\color{incolor}In [{\color{incolor}57}]:} \PY{k}{def} \PY{n+nf}{pengespill}\PY{p}{(}\PY{p}{)}\PY{p}{:}
             \PY{n}{total} \PY{o}{=} \PY{n}{kast\PYZus{}2d6}\PY{p}{(}\PY{p}{)}
             \PY{k}{if} \PY{n}{total} \PY{o}{\PYZlt{}}\PY{o}{=} \PY{l+m+mi}{8}\PY{p}{:}
                 \PY{k}{return} \PY{o}{\PYZhy{}}\PY{l+m+mi}{10}
             \PY{k}{elif} \PY{n}{total} \PY{o}{\PYZlt{}}\PY{o}{=} \PY{l+m+mi}{11}\PY{p}{:}
                 \PY{k}{return} \PY{l+m+mi}{10}
             \PY{k}{else}\PY{p}{:}
                 \PY{k}{return} \PY{l+m+mi}{90}
\end{Verbatim}


    \begin{Verbatim}[commandchars=\\\{\}]
{\color{incolor}In [{\color{incolor}60}]:} \PY{n}{pengespill}\PY{p}{(}\PY{p}{)}
\end{Verbatim}


\begin{Verbatim}[commandchars=\\\{\}]
{\color{outcolor}Out[{\color{outcolor}60}]:} -10
\end{Verbatim}
            
    Hvis vi bare spiller spillet én gang finner vi ikke ut om det er
lønnsomt eller ikke. Enten vinner vi, eller så taper vi, det er
tilfeldig og forteller oss ikke så mye. For å skjønne om det lønner seg
eller ikke må vi spille spillet veldig mange ganger. Istad holdt vi styr
på antall ganger vi vant, men nå kan vi vinne enten litt penger, eller
mye penger, så istedet holder vi styr på den totale fortjenesten (eller
tapet) vårt.

    \begin{Verbatim}[commandchars=\\\{\}]
{\color{incolor}In [{\color{incolor}59}]:} \PY{c+c1}{\PYZsh{} Tellevariabel}
         \PY{n}{fortjeneste} \PY{o}{=} \PY{l+m+mi}{0}
         
         \PY{c+c1}{\PYZsh{} Gjenta spillet 1000 ganger}
         \PY{k}{for} \PY{n}{spill} \PY{o+ow}{in} \PY{n+nb}{range}\PY{p}{(}\PY{l+m+mi}{1000}\PY{p}{)}\PY{p}{:}
             \PY{n}{fortjeneste} \PY{o}{+}\PY{o}{=} \PY{n}{pengespill}\PY{p}{(}\PY{p}{)}
             
         \PY{n+nb}{print}\PY{p}{(}\PY{n}{fortjeneste}\PY{p}{)}
\end{Verbatim}


    \begin{Verbatim}[commandchars=\\\{\}]
-2680

    \end{Verbatim}

    Når vi kjører disse simuleringene ser vi at det er overveldende
sannsynlighet for at vi taper masse penger. Om du kjører det nok ganger
kan du nok til slutt treffe på en kjøring som går i pluss, men
sannsynligheten er veldig lav! Dette spillet er altså ikke lønnsomt selv
om man kan være heldig og vinne en hundrelapp i ny og ne.

    \paragraph{Plotte pengene over tid}\label{plotte-pengene-over-tid}

La oss si vi begynner med 500 kroner, og så spiller vi terningspillet vi
nettopp lagde helt til vi er tom for penger. Dette kan vi også simulere,
la oss vise dem frem som et plott. Som istad bruker vi igjen en
tellevariabel for pengene, men vi lager også en liste som husker summen
etter hvert spill. Det er denne lista vi til slutt plotter.

    \begin{Verbatim}[commandchars=\\\{\}]
{\color{incolor}In [{\color{incolor}70}]:} \PY{c+c1}{\PYZsh{} Telle variabel}
         \PY{n}{penger} \PY{o}{=} \PY{l+m+mi}{500}
         
         \PY{c+c1}{\PYZsh{} Liste for å huske resultatene over tid}
         \PY{n}{pengehistorikk} \PY{o}{=} \PY{p}{[}\PY{p}{]}
         \PY{n}{pengehistorikk}\PY{o}{.}\PY{n}{append}\PY{p}{(}\PY{n}{penger}\PY{p}{)}
         
         \PY{c+c1}{\PYZsh{} Løkke for å gjenta spillet helt til vi går tom for penger}
         \PY{k}{while} \PY{n}{penger} \PY{o}{\PYZgt{}} \PY{l+m+mi}{0}\PY{p}{:}
             \PY{n}{penger} \PY{o}{+}\PY{o}{=} \PY{n}{pengespill}\PY{p}{(}\PY{p}{)}
             \PY{n}{pengehistorikk}\PY{o}{.}\PY{n}{append}\PY{p}{(}\PY{n}{penger}\PY{p}{)}
         
         \PY{c+c1}{\PYZsh{} Plot resultatet}
         \PY{n}{plot}\PY{p}{(}\PY{n}{pengehistorikk}\PY{p}{)}
         \PY{n}{axhline}\PY{p}{(}\PY{l+m+mi}{500}\PY{p}{,} \PY{n}{color}\PY{o}{=}\PY{l+s+s1}{\PYZsq{}}\PY{l+s+s1}{black}\PY{l+s+s1}{\PYZsq{}}\PY{p}{,} \PY{n}{linestyle}\PY{o}{=}\PY{l+s+s1}{\PYZsq{}}\PY{l+s+s1}{\PYZhy{}\PYZhy{}}\PY{l+s+s1}{\PYZsq{}}\PY{p}{)}
         \PY{n}{show}\PY{p}{(}\PY{p}{)}
         \PY{n+nb}{print}\PY{p}{(}\PY{l+s+s2}{\PYZdq{}}\PY{l+s+s2}{Du spilte }\PY{l+s+si}{\PYZob{}\PYZcb{}}\PY{l+s+s2}{ ganger før du gikk tom for penger}\PY{l+s+s2}{\PYZdq{}}\PY{o}{.}\PY{n}{format}\PY{p}{(}\PY{n+nb}{len}\PY{p}{(}\PY{n}{pengehistorikk}\PY{p}{)}\PY{p}{)}\PY{p}{)}
\end{Verbatim}


    \begin{center}
    \adjustimage{max size={0.9\linewidth}{0.9\paperheight}}{output_54_0.png}
    \end{center}
    { \hspace*{\fill} \\}
    
    \begin{Verbatim}[commandchars=\\\{\}]
Du spilte 167 ganger før du gikk tom for penger

    \end{Verbatim}

    Når du kjører programmet ser du fra plottet at det er veldig forskjellig
hvor lang tid det tar før du går tom for penger. Noen ganger spiller du
kanskje 100 ganger, mens andre ganger kanskje 300. Til slutt går du
alltid tom for penger. Den stipla linja er pengene vi begynner med, om
den blå kurven går over den stipla linja betyr det at vi går i pluss.
Noen kjøringer ser vi at vi kanskje går en del i pluss - og om man
slutter å spillet akkurat det er det faktisk mulig å gå i pluss totalt
sett, men som vi skjer er det ikke alltid slik. Det er også vanskelig å
vite når man skal slutte.

    \subsection{Eksempel: Gjettespill}\label{eksempel-gjettespill}

Nå som vi har sett litt på hvordan vi kan generere tilfeldige tall, la
oss lage et lite spill. Vi lar først trekke et tilfeldig tall fra 1 til
1000, men den sier det ikke til brukeren. Så må brukeren gjette på
tallet, og får beskjed om de gjettet for høyt eller for lavt etter hvert
gjett. Programmet holder styr på hvor mange gjett brukeren bruker før de
kommer frem til riktig tall.

For å få brukeren til å gjette bruker vi \texttt{input}-kommandoen, da
kan brukeren skrive inn. Det som er litt viktig da er at det brukeren
skriver vil alltid tolkes som tekst, men vi ønsker å tolke det som tall,
så da må vi også bruke \texttt{int()} for å gjøre om inputten til et
tall.

    \begin{Verbatim}[commandchars=\\\{\}]
{\color{incolor}In [{\color{incolor}24}]:} \PY{n}{fasitsvar} \PY{o}{=} \PY{n}{randint}\PY{p}{(}\PY{l+m+mi}{1}\PY{p}{,} \PY{l+m+mi}{1000}\PY{p}{)}
         \PY{n}{antall\PYZus{}gjett} \PY{o}{=} \PY{l+m+mi}{0}
         
         \PY{n}{gjett} \PY{o}{=} \PY{n+nb}{int}\PY{p}{(}\PY{n+nb}{input}\PY{p}{(}\PY{l+s+s2}{\PYZdq{}}\PY{l+s+s2}{Gjett på et tall mellom 1 og 1000...}\PY{l+s+se}{\PYZbs{}n}\PY{l+s+s2}{\PYZdq{}}\PY{p}{)}\PY{p}{)}
         \PY{n}{antall\PYZus{}gjett} \PY{o}{+}\PY{o}{=} \PY{l+m+mi}{1}
                           
         \PY{k}{while} \PY{n}{gjett} \PY{o}{!=} \PY{n}{fasitsvar}\PY{p}{:}
             \PY{k}{if} \PY{n}{gjett} \PY{o}{\PYZlt{}} \PY{n}{fasitsvar}\PY{p}{:}
                 \PY{n}{gjett} \PY{o}{=} \PY{n+nb}{int}\PY{p}{(}\PY{n+nb}{input}\PY{p}{(}\PY{l+s+s2}{\PYZdq{}}\PY{l+s+s2}{Ditt gjett (}\PY{l+s+si}{\PYZob{}\PYZcb{}}\PY{l+s+s2}{) var for lavt, prøv på nytt...}\PY{l+s+se}{\PYZbs{}n}\PY{l+s+s2}{\PYZdq{}}\PY{o}{.}\PY{n}{format}\PY{p}{(}\PY{n}{gjett}\PY{p}{)}\PY{p}{)}\PY{p}{)}
             \PY{k}{elif} \PY{n}{gjett} \PY{o}{\PYZgt{}} \PY{n}{fasitsvar}\PY{p}{:}
                 \PY{n}{gjett} \PY{o}{=} \PY{n+nb}{int}\PY{p}{(}\PY{n+nb}{input}\PY{p}{(}\PY{l+s+s2}{\PYZdq{}}\PY{l+s+s2}{Ditt gjett (}\PY{l+s+si}{\PYZob{}\PYZcb{}}\PY{l+s+s2}{) var for høyt, prøv på nytt...}\PY{l+s+se}{\PYZbs{}n}\PY{l+s+s2}{\PYZdq{}}\PY{o}{.}\PY{n}{format}\PY{p}{(}\PY{n}{gjett}\PY{p}{)}\PY{p}{)}\PY{p}{)}
             \PY{n}{antall\PYZus{}gjett} \PY{o}{+}\PY{o}{=} \PY{l+m+mi}{1}
                                   
                                   
         \PY{n+nb}{print}\PY{p}{(}\PY{l+s+s2}{\PYZdq{}}\PY{l+s+s2}{Der traff du! }\PY{l+s+si}{\PYZob{}\PYZcb{}}\PY{l+s+s2}{ er riktig svar! Du har brukt }\PY{l+s+si}{\PYZob{}\PYZcb{}}\PY{l+s+s2}{ gjett.}\PY{l+s+s2}{\PYZdq{}}\PY{o}{.}\PY{n}{format}\PY{p}{(}\PY{n}{gjett}\PY{p}{,} \PY{n}{antall\PYZus{}gjett}\PY{p}{)}\PY{p}{)}
\end{Verbatim}


    \begin{Verbatim}[commandchars=\\\{\}]
Gjett på et tall mellom 1 og 1000{\ldots}
500
Ditt gjett (500) var for høyt, prøv på nytt{\ldots}
300
Ditt gjett (300) var for lavt, prøv på nytt{\ldots}
400
Ditt gjett (400) var for lavt, prøv på nytt{\ldots}
450
Ditt gjett (450) var for lavt, prøv på nytt{\ldots}
475
Ditt gjett (475) var for lavt, prøv på nytt{\ldots}
485
Ditt gjett (485) var for lavt, prøv på nytt{\ldots}
495
Ditt gjett (495) var for høyt, prøv på nytt{\ldots}
493
Ditt gjett (493) var for høyt, prøv på nytt{\ldots}
491
Ditt gjett (491) var for høyt, prøv på nytt{\ldots}
490
Der traff du! 490 er riktig svar! Du har brukt 10 gjett.

    \end{Verbatim}

    Her kan du jo tenkte litt over hvilke strategi det er best å bruke for å
komme frem til riktig svar på færrest mulig gjett. Den beste strategien
er \emph{halveringsstrategien}, som vil treffe riktig svar på
\(\log_2(1000)\approx 10\) gjett.

    \subsection{Et par andre
tilfeldighetsfunksjoner}\label{et-par-andre-tilfeldighetsfunksjoner}

Så langt har vi sett hvordan vi kan lage tilfeldige heltall. La oss vise
et par andre tilfeldigheter vi kan lage.

\paragraph{Desimaltall mellom 0 og 1}\label{desimaltall-mellom-0-og-1}

Vi begynner med desimaltall. Om vi kaller på \texttt{rand} får vi et et
tilfeldig desimaltall mellom 0 og 1.

    \begin{Verbatim}[commandchars=\\\{\}]
{\color{incolor}In [{\color{incolor}25}]:} \PY{n+nb}{print}\PY{p}{(}\PY{n}{rand}\PY{p}{(}\PY{p}{)}\PY{p}{)}
         \PY{n+nb}{print}\PY{p}{(}\PY{n}{rand}\PY{p}{(}\PY{p}{)}\PY{p}{)}
         \PY{n+nb}{print}\PY{p}{(}\PY{n}{rand}\PY{p}{(}\PY{p}{)}\PY{p}{)}
\end{Verbatim}


    \begin{Verbatim}[commandchars=\\\{\}]
0.9293872384620315
0.0830080839510704
0.2797391290142661

    \end{Verbatim}

    \texttt{rand()} kan være en veldig nyttig funksjon fordi vi kan bruke
den hver gang det finnes to mulige utfall, uavhengig av hvilke
sannsynligheter de to utfallene har. Si for eksempel at vi ser på et
Pokémon-spill der Pokémons kan konkurere mot hverandre. Når man bruker
et angrep har det en liten sjanse for å være ekstra kraftig, et såkallt
\emph{critical hit}.

 Kilde:
\href{http://pokemon.wikia.com/wiki/File:Generation_III_Critical_Hit.png}{pokemon.wikia.com}

La oss anta det er en 20\% sannsynlighet for å treffe en slik critical
hit, og 80\% sannsynlighet å gjøre et vanlig angrep. Da kan vi bruke
\texttt{rand()} til å sjekke hva slags angrep vi får som følger:

    \begin{Verbatim}[commandchars=\\\{\}]
{\color{incolor}In [{\color{incolor}71}]:} \PY{n}{p} \PY{o}{=} \PY{l+m+mf}{0.20}   \PY{c+c1}{\PYZsh{} Sannsynlighet for å få critical hit}
         \PY{n}{r} \PY{o}{=} \PY{n}{rand}\PY{p}{(}\PY{p}{)} \PY{c+c1}{\PYZsh{} Tilfeldig tall mellom 0 og 1}
         
         \PY{k}{if} \PY{n}{r} \PY{o}{\PYZlt{}} \PY{n}{p}\PY{p}{:}
             \PY{n+nb}{print}\PY{p}{(}\PY{l+s+s2}{\PYZdq{}}\PY{l+s+s2}{A critical hit!}\PY{l+s+s2}{\PYZdq{}}\PY{p}{)}
         \PY{k}{else}\PY{p}{:}
             \PY{n+nb}{print}\PY{p}{(}\PY{l+s+s2}{\PYZdq{}}\PY{l+s+s2}{A normal hit.}\PY{l+s+s2}{\PYZdq{}}\PY{p}{)}
\end{Verbatim}


    \begin{Verbatim}[commandchars=\\\{\}]
A normal hit.

    \end{Verbatim}

    Her setter vi først \texttt{p} til sannsynligheten som et tall mellom 0
og 1(ikke som en prosentverdi). Så trekker vi et tilfeldig tall mellom 0
og 1. Hvis det tilfeldige tallet vi trakk, som vi kaller \texttt{r}, er
mindre enn sannsynligheten, så har vi et ekstra kraftig angrep.

Her tester vi om \(r \leq p\), som vil si at \texttt{r} må være mindre
enn \texttt{p}. Hvorfor er det slik og ikke motsatt? For å forstå dette
kan vi tegne en talllinje fra 0 til 1. Sannsynligheten \texttt{p} deler
denne tallinja i to, i verdier mindre enn \texttt{p}, og i verdier
større enn \texttt{p}. Siden vi sier at \(r < p\) er et treff, farger vi
den venstre siden grønn, og høyresiden rød.

Når vi trekker et tilfeldig tall med \texttt{r\ =\ rand()} kan det havne
hvor som helst på denne tallinja. Jo større \(p\) er, jo større blir
venstresiden, og jo større sannsynlighet får vi for at det tilfeldige
tallet \(r\) havner innenfor det grønne omerådet. I det mest ekstreme er
sannsynligheten 100\%, da blir \(p=1.0\), og alle tilfeldige tall vi
trekker gir \(r < p\).

Denne metoden lar oss altså sjekke om en tilfeldighet intreffer eller
ikke, uansett hvilken sannsynlighet det har for å skje.

    \paragraph{Desimaltall fra andre
intervaller}\label{desimaltall-fra-andre-intervaller}

Noen ganger ønsker vi kanskje desimaltall, men ikke nødvendigvis fra 0
til 1. Kanskje fra -1 til 1, eller fra 0 til 100, eller lignende. Dette
kan vi få til med funksjonen \texttt{uniform(a,\ b)}, som gir et
uniformt fordelt (det betyr at alle verdier er like sannsynlige) som
ligger mellom \texttt{a} og \texttt{b}.

    \textbf{Eksempel:} La oss si vi lager et spill der man skal drive en
dyrehage, alle dyrene vil ha en gitt vekt, som vil øke etterhvert som
dyrene vokser og spiser. Når det blir født dyr skal de genereres med en
tilfelig fordelt startvekt, som bestemmer hvor mye ekstra stell de
trenger og hvor fort de vokser. Et googlesøk forteller oss at en
løvebaby veier typisk 1.5 kg når den blir født, men vi vil ha en del
variabilitet, så vi sier de skal veie mellom 0.75 og 3 kg i spillet
vårt. Da gjør vi som følger

    \begin{Verbatim}[commandchars=\\\{\}]
{\color{incolor}In [{\color{incolor}95}]:} \PY{n}{fødselsvekt} \PY{o}{=} \PY{n}{uniform}\PY{p}{(}\PY{l+m+mf}{0.75}\PY{p}{,} \PY{l+m+mf}{3.0}\PY{p}{)}
         \PY{n+nb}{print}\PY{p}{(}\PY{l+s+s2}{\PYZdq{}}\PY{l+s+s2}{Gratulerer! En ny løve er blitt født i parken din! Fødselsvekten er }\PY{l+s+si}{\PYZob{}:.2f\PYZcb{}}\PY{l+s+s2}{ kg.}\PY{l+s+s2}{\PYZdq{}}\PY{o}{.}\PY{n}{format}\PY{p}{(}\PY{n}{fødselsvekt}\PY{p}{)}\PY{p}{)}
\end{Verbatim}


    \begin{Verbatim}[commandchars=\\\{\}]
Gratulerer! En ny løve er blitt født i parken din! Fødselsvekten er 0.98 kg.

    \end{Verbatim}

    Kanskje vi også vil skrive ut en advarsel til brukeren om en løvepus
veier veldig lite når den blir født

    \begin{Verbatim}[commandchars=\\\{\}]
{\color{incolor}In [{\color{incolor}98}]:} \PY{k}{if} \PY{n}{fødselsvekt} \PY{o}{\PYZlt{}} \PY{l+m+mf}{1.0}\PY{p}{:}
             \PY{n+nb}{print}\PY{p}{(}\PY{l+s+s2}{\PYZdq{}}\PY{l+s+s2}{Advarsel: Den nye løveungen har en unormalt lav fødselsvekt og kan trenge ekstra oppfølgning fra vetrinæren!}\PY{l+s+s2}{\PYZdq{}}\PY{p}{)}
\end{Verbatim}


    \begin{Verbatim}[commandchars=\\\{\}]
Advarsel: Den nye løveungen har en unormalt lav fødselsvekt og kan trenge ekstra oppfølgning fra vetrinæren!

    \end{Verbatim}

    \subsubsection{Å gjøre utvalg med og uten
tilbakelegging}\label{uxe5-gjuxf8re-utvalg-med-og-uten-tilbakelegging}

La oss si vi har 5 personer, og vi skal plukke ut en heldig vinner som
skal få kinobilletter. Dette ville vi vanligvis gjort ved å trekke lodd.
For å gjøre dette i Python lager vi først en liste over de 5 personene,
så bruker vi funksjonen \texttt{choice}, som trekker et tilfeldig
element fra en liste

    \begin{Verbatim}[commandchars=\\\{\}]
{\color{incolor}In [{\color{incolor}100}]:} \PY{n}{personer} \PY{o}{=} \PY{p}{[}\PY{l+s+s2}{\PYZdq{}}\PY{l+s+s2}{Anders}\PY{l+s+s2}{\PYZdq{}}\PY{p}{,} \PY{l+s+s2}{\PYZdq{}}\PY{l+s+s2}{Beate}\PY{l+s+s2}{\PYZdq{}}\PY{p}{,} \PY{l+s+s2}{\PYZdq{}}\PY{l+s+s2}{Christine}\PY{l+s+s2}{\PYZdq{}}\PY{p}{,} \PY{l+s+s2}{\PYZdq{}}\PY{l+s+s2}{Daniel}\PY{l+s+s2}{\PYZdq{}}\PY{p}{,} \PY{l+s+s2}{\PYZdq{}}\PY{l+s+s2}{Erika}\PY{l+s+s2}{\PYZdq{}}\PY{p}{]}
          \PY{n}{vinner} \PY{o}{=} \PY{n}{choice}\PY{p}{(}\PY{n}{personer}\PY{p}{)}
          \PY{n+nb}{print}\PY{p}{(}\PY{n}{vinner}\PY{p}{)}
\end{Verbatim}


    \begin{Verbatim}[commandchars=\\\{\}]
Daniel

    \end{Verbatim}

    \emph{Choice} betyr "valg", på engelsk, og det funksjonen er å velge ett
element tilfeldig. Dette kaller vi gjerne et \emph{utvalg}.

    Kanskje vi har to premier å gi bort, da ønsker vi å velge to vinnere og
gjør dette ved å sende antallet som et ekstra argument:
\texttt{choice(personer,\ 2)}. Når vi trekker flere elementer fra en
mengde kan vi gjøre dette \emph{med} eller \emph{uten} tilbakelegging.
Tenk deg at vi har en bolle med en papirlapp med hvert navn på. Om vi
trekker en lapp, og ikke legger den tilbake før vi trekker neste, så kan
ikke én person vinne begge premiene. Om vi derimot legger tilbake lappen
før vi trekker igjen, så er det mulig at vi treffer samme lapp begge
ganger.

Funksjonen vi bruker, \texttt{choice}, er vanligvis utvalg \emph{med}
tilbakelegging.

    \begin{Verbatim}[commandchars=\\\{\}]
{\color{incolor}In [{\color{incolor}101}]:} \PY{n}{vinnere} \PY{o}{=} \PY{n}{choice}\PY{p}{(}\PY{n}{personer}\PY{p}{,} \PY{l+m+mi}{2}\PY{p}{)}
          \PY{n+nb}{print}\PY{p}{(}\PY{n}{vinnere}\PY{p}{)}
\end{Verbatim}


    \begin{Verbatim}[commandchars=\\\{\}]
['Christine' 'Christine']

    \end{Verbatim}

    Fordi \texttt{choice} er med tilbakelegging vil du kunne få et resultat
hvor begge premiene går til samme person. Om vi ønsker å unngå dette må
vi trekke \emph{uten tilbakelegging}. Funksjonen \texttt{choice} kan få
til dette også, men da må vi settet \texttt{replace} til falskt. Replace
er engelsk å betyr å erstatte eller legge tilbake, så
\texttt{replace=False} betyr at vi \emph{ikke vil ha tilbakelegging}.

    \begin{Verbatim}[commandchars=\\\{\}]
{\color{incolor}In [{\color{incolor}102}]:} \PY{n}{vinnere} \PY{o}{=} \PY{n}{choice}\PY{p}{(}\PY{n}{personer}\PY{p}{,} \PY{l+m+mi}{2}\PY{p}{,} \PY{n}{replace}\PY{o}{=}\PY{k+kc}{False}\PY{p}{)}
          \PY{n+nb}{print}\PY{p}{(}\PY{n}{vinnere}\PY{p}{)}
\end{Verbatim}


    \begin{Verbatim}[commandchars=\\\{\}]
['Anders' 'Daniel']

    \end{Verbatim}

    \textbf{Tenkespørsmål:} Hvis du prøver \texttt{choice(personer,\ 6)}
fungerer dette fint, men \texttt{choice(personer,\ 6,\ replace=False)}
fungerer ikke. Hvorfor fungerer den ene, men ikke den andre?

    \textbf{Eksempeloppgave}

Vi har en pose med 20 røde baller og 30 blå baller. Hvis vi trekker 10
baller fra sekken, uten tilbakelegging, hva er sannsynligheten for å
trekke minst to baller av hver farge?

    \begin{Verbatim}[commandchars=\\\{\}]
{\color{incolor}In [{\color{incolor}32}]:} \PY{n}{antall\PYZus{}simuleringer} \PY{o}{=} \PY{l+m+mi}{1000}
         \PY{n}{baller} \PY{o}{=} \PY{p}{[}\PY{l+s+s2}{\PYZdq{}}\PY{l+s+s2}{rød}\PY{l+s+s2}{\PYZdq{}}\PY{p}{]}\PY{o}{*}\PY{l+m+mi}{20} \PY{o}{+} \PY{p}{[}\PY{l+s+s2}{\PYZdq{}}\PY{l+s+s2}{blå}\PY{l+s+s2}{\PYZdq{}}\PY{p}{]}\PY{o}{*}\PY{l+m+mi}{30}
         
         \PY{n}{gunstige\PYZus{}simuleringer} \PY{o}{=} \PY{l+m+mi}{0}
         \PY{k}{for} \PY{n}{trekkning} \PY{o+ow}{in} \PY{n+nb}{range}\PY{p}{(}\PY{n}{antall\PYZus{}simuleringer}\PY{p}{)}\PY{p}{:}
             \PY{n}{utvalg} \PY{o}{=} \PY{n}{choice}\PY{p}{(}\PY{n}{baller}\PY{p}{,} \PY{l+m+mi}{10}\PY{p}{,} \PY{n}{replace}\PY{o}{=}\PY{k+kc}{False}\PY{p}{)}
             
             \PY{c+c1}{\PYZsh{} Telle antall av hver farge}
             \PY{n}{røde\PYZus{}baller} \PY{o}{=} \PY{l+m+mi}{0}
             \PY{n}{blåe\PYZus{}baller} \PY{o}{=} \PY{l+m+mi}{0}
             \PY{k}{for} \PY{n}{ball} \PY{o+ow}{in} \PY{n}{utvalg}\PY{p}{:}
                 \PY{k}{if} \PY{n}{ball} \PY{o}{==} \PY{l+s+s2}{\PYZdq{}}\PY{l+s+s2}{rød}\PY{l+s+s2}{\PYZdq{}}\PY{p}{:}
                     \PY{n}{røde\PYZus{}baller} \PY{o}{+}\PY{o}{=} \PY{l+m+mi}{1}
                 \PY{k}{else}\PY{p}{:}
                     \PY{n}{blåe\PYZus{}baller} \PY{o}{+}\PY{o}{=} \PY{l+m+mi}{1}
             
             \PY{c+c1}{\PYZsh{} Sjekke om vi har minst 3 av hver}
             \PY{k}{if} \PY{n}{røde\PYZus{}baller} \PY{o}{\PYZgt{}}\PY{o}{=} \PY{l+m+mi}{3} \PY{o+ow}{and} \PY{n}{blåe\PYZus{}baller} \PY{o}{\PYZgt{}}\PY{o}{=} \PY{l+m+mi}{3}\PY{p}{:}
                 \PY{n}{gunstige\PYZus{}simuleringer} \PY{o}{+}\PY{o}{=} \PY{l+m+mi}{1}
             
         \PY{n+nb}{print}\PY{p}{(}\PY{l+s+s2}{\PYZdq{}}\PY{l+s+si}{\PYZob{}:.1\PYZpc{}\PYZcb{}}\PY{l+s+s2}{\PYZdq{}}\PY{o}{.}\PY{n}{format}\PY{p}{(}\PY{n}{gunstige\PYZus{}simuleringer}\PY{o}{/}\PY{n}{antall\PYZus{}simuleringer}\PY{p}{)}\PY{p}{)}
\end{Verbatim}


    \begin{Verbatim}[commandchars=\\\{\}]
85.2\%

    \end{Verbatim}

    \subsubsection{Stokke en kortstokk}\label{stokke-en-kortstokk}

La oss si vi vil lage et kortspill på datamaskin. Da trenger vi en måte
å stokke kortstokken, heldigvis finnes dette, funksjonen
\texttt{shuffle()} tar en liste og stokker om på den, så rekkefølgen er
helt tilfeldig.

La oss først lage en kortstokk. Vi lar en kortstokk være en liste der
hvert element er et kort. Et kort i en kortstokk har to egenskaper:
farge (fargene er ruter/hjerter/spar/kløver, ikke rød/svart) og valør
(tallverdi). Istedet for å skrive ut alle 52 kortene i en kortstokk kan
vi bruke \emph{to} løkker for å lage en kortstokk som følger:

    \begin{Verbatim}[commandchars=\\\{\}]
{\color{incolor}In [{\color{incolor}110}]:} \PY{c+c1}{\PYZsh{} Lager en tom stokk}
          \PY{n}{kortstokk} \PY{o}{=} \PY{p}{[}\PY{p}{]}
          
          \PY{c+c1}{\PYZsh{} Løkke over alle farger og valører. Det skal finnes nøyaktig ett kort per kombinasjon av valør og farge}
          \PY{k}{for} \PY{n}{farge} \PY{o+ow}{in} \PY{p}{(}\PY{l+s+s2}{\PYZdq{}}\PY{l+s+s2}{Ruter}\PY{l+s+s2}{\PYZdq{}}\PY{p}{,} \PY{l+s+s2}{\PYZdq{}}\PY{l+s+s2}{Hjerter}\PY{l+s+s2}{\PYZdq{}}\PY{p}{,} \PY{l+s+s2}{\PYZdq{}}\PY{l+s+s2}{Spar}\PY{l+s+s2}{\PYZdq{}}\PY{p}{,} \PY{l+s+s2}{\PYZdq{}}\PY{l+s+s2}{Kløver}\PY{l+s+s2}{\PYZdq{}}\PY{p}{)}\PY{p}{:}
              \PY{k}{for} \PY{n}{valør} \PY{o+ow}{in} \PY{p}{(}\PY{l+m+mi}{2}\PY{p}{,} \PY{l+m+mi}{3}\PY{p}{,} \PY{l+m+mi}{4}\PY{p}{,} \PY{l+m+mi}{5}\PY{p}{,} \PY{l+m+mi}{6}\PY{p}{,} \PY{l+m+mi}{7}\PY{p}{,} \PY{l+m+mi}{8}\PY{p}{,} \PY{l+m+mi}{9}\PY{p}{,} \PY{l+m+mi}{10}\PY{p}{,} \PY{l+s+s2}{\PYZdq{}}\PY{l+s+s2}{Knekt}\PY{l+s+s2}{\PYZdq{}}\PY{p}{,} \PY{l+s+s2}{\PYZdq{}}\PY{l+s+s2}{Dame}\PY{l+s+s2}{\PYZdq{}}\PY{p}{,} \PY{l+s+s2}{\PYZdq{}}\PY{l+s+s2}{Konge}\PY{l+s+s2}{\PYZdq{}}\PY{p}{,} \PY{l+s+s2}{\PYZdq{}}\PY{l+s+s2}{Ess}\PY{l+s+s2}{\PYZdq{}}\PY{p}{)}\PY{p}{:}
                  \PY{c+c1}{\PYZsh{} Legg til kortet til kortstokken}
                  \PY{n}{kortstokk}\PY{o}{.}\PY{n}{append}\PY{p}{(}\PY{p}{(}\PY{n}{farge}\PY{p}{,} \PY{n}{valør}\PY{p}{)}\PY{p}{)}
          
          \PY{c+c1}{\PYZsh{} Sjekke at det ble 52 kort        }
          \PY{n+nb}{print}\PY{p}{(}\PY{n+nb}{len}\PY{p}{(}\PY{n}{kortstokk}\PY{p}{)}\PY{p}{)}
          
          \PY{c+c1}{\PYZsh{} Skriv ut de første 5 kortene i stokken}
          \PY{k}{for} \PY{n}{i} \PY{o+ow}{in} \PY{n+nb}{range}\PY{p}{(}\PY{l+m+mi}{5}\PY{p}{)}\PY{p}{:}
              \PY{n+nb}{print}\PY{p}{(}\PY{n}{kortstokk}\PY{p}{[}\PY{n}{i}\PY{p}{]}\PY{p}{)}
\end{Verbatim}


    \begin{Verbatim}[commandchars=\\\{\}]
52
('Ruter', 2)
('Ruter', 3)
('Ruter', 4)
('Ruter', 5)
('Ruter', 6)

    \end{Verbatim}

    Vi har altså lagd en kortstokk med 52 kort, men vi ser den er i samme
rekkefølge som vi lagde løkkene våre i. De fleste kortspill ville vært
fryktelig kjedlige om vi ikke stokker kortstokken. Når vi har en liste
med elementer kan vi stokke om på rekkefølgen med \texttt{shuffle}
(shuffle er engelsk og betyr å stokke om på)

    \begin{Verbatim}[commandchars=\\\{\}]
{\color{incolor}In [{\color{incolor}111}]:} \PY{n}{shuffle}\PY{p}{(}\PY{n}{kortstokk}\PY{p}{)}
\end{Verbatim}


    Når vi kaller \texttt{shuffle(kortstokk)} stokkes lista vår om på. Om vi
nå skriver ut de første 5 kortene i lista for eksempel, vil vi se den er
godt stokket

    \begin{Verbatim}[commandchars=\\\{\}]
{\color{incolor}In [{\color{incolor}112}]:} \PY{k}{for} \PY{n}{kort} \PY{o+ow}{in} \PY{n}{kortstokk}\PY{p}{[}\PY{p}{:}\PY{l+m+mi}{5}\PY{p}{]}\PY{p}{:}
              \PY{n+nb}{print}\PY{p}{(}\PY{n}{kort}\PY{p}{)}
\end{Verbatim}


    \begin{Verbatim}[commandchars=\\\{\}]
('Ruter', 9)
('Hjerter', 4)
('Ruter', 4)
('Spar', 3)
('Ruter', 'Knekt')

    \end{Verbatim}

    Nå har vi altså en kortstokk og en måte å stokke den, det er et veldig
godt grunnlag for å lage et kortspill! En nyttig funksjon å bruke kan da
være at \texttt{kortstokk.pop(0)} trekker det øverste (det 0'te) kortet
i kortstokken. Det vil si at kortet returneres fra funksjonskallet, og
kortet fjernes fra kortstokken.

La oss for eksempel si at vi skal lage et pokerspill. I for eksempel
Texas Hold'em får hver spiller 2 kort. La oss si vi har 5 spillere. Da
kan vi dele ut starthender som følger

    \begin{Verbatim}[commandchars=\\\{\}]
{\color{incolor}In [{\color{incolor}114}]:} \PY{c+c1}{\PYZsh{} Opprette tomme hender for å oppbevare kortene som trekkes}
          \PY{n}{hender} \PY{o}{=} \PY{p}{[}\PY{p}{]}
          
          \PY{c+c1}{\PYZsh{} Vi skal dele ut to kort til hver spiller}
          \PY{k}{for} \PY{n}{spiller} \PY{o+ow}{in} \PY{n+nb}{range}\PY{p}{(}\PY{l+m+mi}{5}\PY{p}{)}\PY{p}{:}   
              \PY{n}{hånd} \PY{o}{=} \PY{p}{[}\PY{p}{]}
              \PY{n}{hånd}\PY{o}{.}\PY{n}{append}\PY{p}{(}\PY{n}{kortstokk}\PY{o}{.}\PY{n}{pop}\PY{p}{(}\PY{l+m+mi}{0}\PY{p}{)}\PY{p}{)}
              \PY{n}{hånd}\PY{o}{.}\PY{n}{append}\PY{p}{(}\PY{n}{kortstokk}\PY{o}{.}\PY{n}{pop}\PY{p}{(}\PY{l+m+mi}{0}\PY{p}{)}\PY{p}{)}
              \PY{n}{hender}\PY{o}{.}\PY{n}{append}\PY{p}{(}\PY{n}{hånd}\PY{p}{)}
              
          \PY{c+c1}{\PYZsh{} Skriv ut hendene til slutt}
          \PY{k}{for} \PY{n}{spiller} \PY{o+ow}{in} \PY{n+nb}{range}\PY{p}{(}\PY{l+m+mi}{5}\PY{p}{)}\PY{p}{:}
              \PY{n+nb}{print}\PY{p}{(}\PY{n}{hender}\PY{p}{[}\PY{n}{spiller}\PY{p}{]}\PY{p}{)}
\end{Verbatim}


    \begin{Verbatim}[commandchars=\\\{\}]
[('Ruter', 9), ('Hjerter', 4)]
[('Ruter', 4), ('Spar', 3)]
[('Ruter', 'Knekt'), ('Kløver', 'Dame')]
[('Hjerter', 7), ('Spar', 10)]
[('Kløver', 7), ('Kløver', 8)]

    \end{Verbatim}

    \textbf{Forslag til prosjekt:} Som et programmeringsprosjekt kan man her
utfordre elevene til å lage kortspiller Krig på datamaskin. Dette er et
av de enkleste kortspillene å implementere, spesielt siden det bare er
to spillere. Her kan man først lage et spill der en person spiller mot
datamaskin, eller man kan la to mennesker spille mot hverandre. Et slikt
prosjekt er ganske motiverende for noen elever, som syns det er veldig
morsomt å lage spill på datamaskin. Derimot er det kanskje ikke like mye
faglig relevans i et slikt opplegg, men man kan lære masse god
programmering av det!

    \section{Store Talls lov}\label{store-talls-lov}

Vi skal nå bruke det vi har lært så langt til å se på \emph{store talls
lov}. Store talls lov sier at når antallet tilfeldige forsøk blir stort,
så vil fordelingen nærme seg sannsynligheten. Si vi for eksempel gjør
myntkast med en helt rettferdig mynt, det vil si en som har 50\%
sannsynlighet for å lande på den ene side, og 50\% for å havne på den
andre. Om vi for eksemel kaster 10 ganger på rad, så er det langt fra
umulig å for eksempel få en fordeling på 3/7, som tilsier at
sannsynlighetene er 30\%/70\%! Om vi derimot kaster 1000 ganger, vil det
bli langt vanskligere å få en 30\%/70\% fordeling i store talls lov, for
det ville krevd at vi kastet 300 kron og 700 mynt, som er usannsynlig.
Det er altså vanskligere å havne langt unna de faktiske sannsynlighetene
når vi gjentar forsøket mange ganger.

La oss se på dette, vi bruker myntkast funksjonen vi lagde på starten av
notebooken

    \begin{Verbatim}[commandchars=\\\{\}]
{\color{incolor}In [{\color{incolor}115}]:} \PY{k}{def} \PY{n+nf}{sammenligne\PYZus{}kron\PYZus{}og\PYZus{}mynt}\PY{p}{(}\PY{n}{antall\PYZus{}kast}\PY{p}{)}\PY{p}{:}
              \PY{n}{kron}\PY{p}{,} \PY{n}{mynt} \PY{o}{=} \PY{n}{mange\PYZus{}myntkast}\PY{p}{(}\PY{n}{antall\PYZus{}kast}\PY{p}{)}
              \PY{n+nb}{print}\PY{p}{(}\PY{l+s+s2}{\PYZdq{}}\PY{l+s+si}{\PYZob{}\PYZcb{}}\PY{l+s+s2}{ kron (}\PY{l+s+si}{\PYZob{}:.1\PYZpc{}\PYZcb{}}\PY{l+s+s2}{)  |  }\PY{l+s+si}{\PYZob{}\PYZcb{}}\PY{l+s+s2}{ mynt (}\PY{l+s+si}{\PYZob{}:.1\PYZpc{}\PYZcb{}}\PY{l+s+s2}{).}\PY{l+s+s2}{\PYZdq{}}\PY{o}{.}\PY{n}{format}\PY{p}{(}\PY{n}{kron}\PY{p}{,} \PY{n}{kron}\PY{o}{/}\PY{n}{antall\PYZus{}kast}\PY{p}{,} \PY{n}{mynt}\PY{p}{,} \PY{n}{mynt}\PY{o}{/}\PY{n}{antall\PYZus{}kast}\PY{p}{)}\PY{p}{)}
\end{Verbatim}


    Vi prøver først noen ganger med 10 kast, så noen ganger med 1000, så
noen ganger med 1 million kast. Når vi kaster så mange ganger kan det
være det begynner å gå litt tregt når vi kjører koden, men den kommer i
mål til slutt.

    \begin{Verbatim}[commandchars=\\\{\}]
{\color{incolor}In [{\color{incolor}116}]:} \PY{n}{sammenligne\PYZus{}kron\PYZus{}og\PYZus{}mynt}\PY{p}{(}\PY{l+m+mi}{10}\PY{p}{)}
          \PY{n}{sammenligne\PYZus{}kron\PYZus{}og\PYZus{}mynt}\PY{p}{(}\PY{l+m+mi}{10}\PY{p}{)}
          \PY{n}{sammenligne\PYZus{}kron\PYZus{}og\PYZus{}mynt}\PY{p}{(}\PY{l+m+mi}{10}\PY{p}{)}
          \PY{n}{sammenligne\PYZus{}kron\PYZus{}og\PYZus{}mynt}\PY{p}{(}\PY{l+m+mi}{10}\PY{p}{)}
          \PY{n}{sammenligne\PYZus{}kron\PYZus{}og\PYZus{}mynt}\PY{p}{(}\PY{l+m+mi}{10}\PY{p}{)}
\end{Verbatim}


    \begin{Verbatim}[commandchars=\\\{\}]
6 kron (60.0\%)  |  4 mynt (40.0\%).
5 kron (50.0\%)  |  5 mynt (50.0\%).
5 kron (50.0\%)  |  5 mynt (50.0\%).
3 kron (30.0\%)  |  7 mynt (70.0\%).
7 kron (70.0\%)  |  3 mynt (30.0\%).

    \end{Verbatim}

    \begin{Verbatim}[commandchars=\\\{\}]
{\color{incolor}In [{\color{incolor}117}]:} \PY{n}{sammenligne\PYZus{}kron\PYZus{}og\PYZus{}mynt}\PY{p}{(}\PY{l+m+mi}{1000}\PY{p}{)}
          \PY{n}{sammenligne\PYZus{}kron\PYZus{}og\PYZus{}mynt}\PY{p}{(}\PY{l+m+mi}{1000}\PY{p}{)}
          \PY{n}{sammenligne\PYZus{}kron\PYZus{}og\PYZus{}mynt}\PY{p}{(}\PY{l+m+mi}{1000}\PY{p}{)}
          \PY{n}{sammenligne\PYZus{}kron\PYZus{}og\PYZus{}mynt}\PY{p}{(}\PY{l+m+mi}{1000}\PY{p}{)}
          \PY{n}{sammenligne\PYZus{}kron\PYZus{}og\PYZus{}mynt}\PY{p}{(}\PY{l+m+mi}{1000}\PY{p}{)}
\end{Verbatim}


    \begin{Verbatim}[commandchars=\\\{\}]
492 kron (49.2\%)  |  508 mynt (50.8\%).
505 kron (50.5\%)  |  495 mynt (49.5\%).
487 kron (48.7\%)  |  513 mynt (51.3\%).
495 kron (49.5\%)  |  505 mynt (50.5\%).
487 kron (48.7\%)  |  513 mynt (51.3\%).

    \end{Verbatim}

    \begin{Verbatim}[commandchars=\\\{\}]
{\color{incolor}In [{\color{incolor}118}]:} \PY{n}{sammenligne\PYZus{}kron\PYZus{}og\PYZus{}mynt}\PY{p}{(}\PY{l+m+mi}{1000000}\PY{p}{)}
          \PY{n}{sammenligne\PYZus{}kron\PYZus{}og\PYZus{}mynt}\PY{p}{(}\PY{l+m+mi}{1000000}\PY{p}{)}
          \PY{n}{sammenligne\PYZus{}kron\PYZus{}og\PYZus{}mynt}\PY{p}{(}\PY{l+m+mi}{1000000}\PY{p}{)}
          \PY{n}{sammenligne\PYZus{}kron\PYZus{}og\PYZus{}mynt}\PY{p}{(}\PY{l+m+mi}{1000000}\PY{p}{)}
          \PY{n}{sammenligne\PYZus{}kron\PYZus{}og\PYZus{}mynt}\PY{p}{(}\PY{l+m+mi}{1000000}\PY{p}{)}
\end{Verbatim}


    \begin{Verbatim}[commandchars=\\\{\}]
500519 kron (50.1\%)  |  499481 mynt (49.9\%).
499506 kron (50.0\%)  |  500494 mynt (50.0\%).
500807 kron (50.1\%)  |  499193 mynt (49.9\%).
500334 kron (50.0\%)  |  499666 mynt (50.0\%).
499619 kron (50.0\%)  |  500381 mynt (50.0\%).

    \end{Verbatim}

    Vi ser at jo flere kast vi gjør, jo sikrere blir vi på at utfallet
ligger nærme den faktiske sannsynligheten på 50\%. Det er dette som er
store talls lov.

    \paragraph{En mynt har ingen
hukomelse}\label{en-mynt-har-ingen-hukomelse}

Hvor kommer store talls lov fra? Hva er det som gjør at det i det lange
løp vil utligne seg på 50\%? Dette er et spørsmål det er vanskelig å
svare på, for oss mennesker er ikke så veldig flinke på det å tenkte på
sannsynligheter og tilfeldighet.

En veldig vanlig ting å tenke, er at mynten \emph{husker} hva som har
skjedd. Om vi for eksempel driver med myntkast, og har fått 5 kron på
rad, er det veldig lett å tenke at det nesten må være mer sannsynlig at
neste kast blir mynt, for vi må jo veie opp for at det ble så mye kron.
Men sånn er det ikke! En mynt har ingen hukomelse, og hver gang vi kaser
er det 50/50\% for å få kron og mynt, selv om vi kanskje har nettopp
kastet 10 kron på rad, er det like sannsynlig å få kron på neste kast
som en mynt.

Dette gjelder også for terningkast. Om man har kastet en terning 20
ganger uten å få en eneste 6'er tenker man fort nå \textbf{må} det jo
komme en 6-er snart, men terningen har heller ingen hukkomelse, og selv
etter en lang periode uten en 6'er, er det akkurat like sannsynlig å
rulle de andre tallene på terningen.

Denne tankegangen er altså feil, men utrolig vanlig, og den har fått
navnet \emph{Gambler's fallacy} på engelsk, en \emph{fallacy} er en
tankefeil eller feilslutning, så det kan oversettes som Gamblerens
tankefeil.

\paragraph{Monte Carlo i 1913}\label{monte-carlo-i-1913}

(Kilde: Du kan lese mer om denne historien for eksempel i
\href{http://www.bbc.com/future/story/20150127-why-we-gamble-like-monkeys}{denne
BBC artikkelen}.)

Det er et veldig kjent eksempel av Gamblerens tankefeil fra et casio i
Monte Carlo i 1913. Ved et rulettbord rulles en liten kule rundt i et
ruletthjul, og havner tilfeldig på et tall fra 0 til 36. Med unntak av
0, som er grønn, er det like mange rød og svarte felter. Man kan sette
penger på om kula lander på rød eller svart på neste spinn. Det som
skjedde i 1913 var at et av Casionets ruletthjul begynnte å treffe
veldig mange svare på rad.

 (Kilde:
\href{https://www.flickr.com/photos/dahlstroms/5276348473}{Håkan
Dahlström} under \href{https://creativecommons.org/licenses/by/2.0/}{CC
BY 2.0})

Etterhvert som det kom fler og fler svarte på rad begynnte fler og fler
å samle seg rundt hjulet og mange trodde at det \emph{måtte} dukke opp
en rød snart! Folk var så sikre på dette at de var villige til å sette
alle pengene sine på rød, noen satset flere millioner! Det kom hele 26
svarte baller på rad, før det til slutt kom en rød ball. Innen den røde
ballen dukket opp hadde de fleste allerede spilt fra seg alle pengene
sine. De ble lurt av magefølelsen til å tro at den rød ballen var mer
sannsynlig, men det er den ikke. Et rulletthjul, akkurat som en mynt,
har ingen hukkomelse.

    \subsubsection{Hvorfor fungerer Store Talls
Lov?}\label{hvorfor-fungerer-store-talls-lov}

Men hvis en mynt, eller et rulletthjul, ikke har noen hukkomelse -
hvorfor virker store talls lov? Om hjulet havnet på svart 26 ganger på
rad, men det ikke kommer noen flere røde etterpå, vil det ikke da være
flere svarte enn røde baller til slutt, altså ikke en 50/50 fordeling
slik store talls lov sier?

For å forstå dette er det viktig å lese veldig nøye hva store talls lov
sier. Store talls lov sier at forholdet vil nærme seg
\emph{sannsynligheten}. Vi forventer altså at vi får 50\% mynt og 50\%
kron. Dette er \emph{ikke} det samme som å si at vi skal få like mange
kron og mynt! "Hæ ?" Sier du kanskje nå, hvis vi får 50\% mynt og 50\%
kron, hvordan kan det ikke være like mange av dem?

Hvis du går tilbake til der du flippet først 10 mynter, så 1000, så
1000000. Se på \emph{avviket} fra perfekt fordeling. Når vi flippet 10
mynter ser vi avvik på 0-3. Det vil si at vi forventer en fordeling på 5
kron og 5 mynt, men det skjer av og til at vi f.eks får 8 kron og 2
mynt. Om vi ser på avvikene vi får når vi flipper 1000 mynter ser vi at
disse er på ca 20. Som vil si at vi forventer å få ca 500/500, men vi
ser vi kan lett får 480 kron og 520 mynt. For en million kast ser vi
avvik på 500-1000 fra den perfekte fordelingen.

Vi ser altså at når vi kaster fler og fler mynter så vil avviket fra den
perfekte fordelingen bli større og større. Jo fler mynter du kaster, jo
mindre sannsynlig er det å få nøyaktig like mange kron og mynt. Men, når
du ser på sannsynligheten, så er historien helt annerledes. Vi ser at
sannsynligheten blir nærmere og nærmere 50\%! Hva er det som foregår
her?

    \paragraph{Litt Matematikk for å forklare
oppførselen}\label{litt-matematikk-for-uxe5-forklare-oppfuxf8rselen}

(Denne forklaringen er nok litt voldsom for ungdomsskolen, men du får
gjøre en vurdering på hvor i detalj dere vil gå i klasserommet eller om
du vil at tallene fra simuleringene snakker litt for seg selv!)

La oss gi kron verdien -1, og mynt verdien 1. (Du kan tenke deg at du
spiller om 1 kron per kast). Siden vi forventer å få like mange av dem,
så kommer den totale fortjenesten til å ligge på 0 kroner. Men som vi
har sett vil avviket fra denne perfekte fordelingen (akkurat 0 kroner)
øke over tid. Vi kan vise, ved hjelp av litt fancy matematikk, at vi
kommer til å forvente et avvik på opp til \(\sqrt{n}\) etter \(n\) kast.
Så for 10, 1000 og 1000000 kast vil vi typisk forvente å se utslag på
\[\begin{align*}
\sqrt{10} &\approx 3 \\
\sqrt{1000} &\approx 32 \\
\sqrt{1000000} &= 1000
\end{align*}
\] Så etter en million kast kan vi fort ligge opptil 1000 kroner i pluss
eller i minus, men noe særlig mer enn dette er veldig usannsynlig.

Men \emph{sannsynligheten} er gitt ved å dele på antall kast, så hvis vi
forventer et avvik på opptil 3 kast, betyr dette vi har et avvik på
\(\frac{3}{10} \approx 0.3\) i sannsynligheten, hele 30\%! For 1000 kast
derimot, har vi et avvik på opptil 32, som gir et avvik på opptil
\(\frac{32}{1000} \approx 0.032\), det vil si 3.2\%. Mye lavere. Til
slutt har vi en million kast, som gir oss avik på
\(1000/1000000 = 0.001\), så lite som 0.1\%!

For å ta det på et litt avansert nivå: Helt generelt får vi at avviket i
antall kron og mynt er \(\sqrt{n}\), som vokser med antall kast, mens
avviket i sannsynligheten vil være opptil \(\sqrt{n}/n = 1/\sqrt{n}\).
Og siden \(\sqrt{n}\) øker med antall kast \(n\), så vil \(1/\sqrt{n}\)
minske.

    \subsubsection{Plotte Store talls lov}\label{plotte-store-talls-lov}

Vi kommer nå til å simulere myntkast igjen, men denne gangen skal vi
plotte resultatene. Denne biten viser vi bare som eksempel, det blir nok
litt mye for elevene å skjønne. Her kan du for eksempel kjøre dette på
din egen maskin, og så vise resultatene og diskutere og tolke det sammen
med elevene.

Vi kaster nå mynt mange ganger, og lar mynt ha verdi -1 og kron 1. Vi
plotter både totalverdien vi har over tid, men også
sannsynligheten/fordelingen av kron og mynt vi har fått.

    \begin{Verbatim}[commandchars=\\\{\}]
{\color{incolor}In [{\color{incolor}176}]:} \PY{n}{antall\PYZus{}kast} \PY{o}{=} \PY{l+m+mi}{1000}
          \PY{n}{verdi} \PY{o}{=} \PY{l+m+mi}{0}
          \PY{n}{historikk} \PY{o}{=} \PY{p}{[}\PY{p}{]}
          
          \PY{k}{for} \PY{n}{kast} \PY{o+ow}{in} \PY{n+nb}{range}\PY{p}{(}\PY{n}{antall\PYZus{}kast}\PY{p}{)}\PY{p}{:}
              \PY{n}{verdi} \PY{o}{+}\PY{o}{=} \PY{l+m+mi}{2}\PY{o}{*}\PY{n}{randint}\PY{p}{(}\PY{l+m+mi}{2}\PY{p}{)} \PY{o}{\PYZhy{}} \PY{l+m+mi}{1} \PY{c+c1}{\PYZsh{} Vi gjør dette for å får \PYZhy{}1 eller 1, istedenfor 0 eller 1}
              
              \PY{c+c1}{\PYZsh{} Ta vare på verdiene for plotting}
              \PY{n}{historikk}\PY{o}{.}\PY{n}{append}\PY{p}{(}\PY{n}{verdi}\PY{p}{)}
              
          \PY{n}{plot}\PY{p}{(}\PY{n}{historikk}\PY{p}{)}
          \PY{n}{xlabel}\PY{p}{(}\PY{l+s+s1}{\PYZsq{}}\PY{l+s+s1}{Antall Kast}\PY{l+s+s1}{\PYZsq{}}\PY{p}{)}
          \PY{n}{ylabel}\PY{p}{(}\PY{l+s+s1}{\PYZsq{}}\PY{l+s+s1}{Avvik fra perfekt fordeling}\PY{l+s+s1}{\PYZsq{}}\PY{p}{)}
          \PY{n}{axhline}\PY{p}{(}\PY{l+m+mi}{0}\PY{p}{,} \PY{n}{linestyle}\PY{o}{=}\PY{l+s+s1}{\PYZsq{}}\PY{l+s+s1}{\PYZhy{}\PYZhy{}}\PY{l+s+s1}{\PYZsq{}}\PY{p}{,} \PY{n}{color}\PY{o}{=}\PY{l+s+s1}{\PYZsq{}}\PY{l+s+s1}{black}\PY{l+s+s1}{\PYZsq{}}\PY{p}{)}
          \PY{n}{axis}\PY{p}{(}\PY{p}{(}\PY{l+m+mi}{0}\PY{p}{,} \PY{l+m+mi}{1000}\PY{p}{,} \PY{o}{\PYZhy{}}\PY{l+m+mi}{75}\PY{p}{,} \PY{l+m+mi}{75}\PY{p}{)}\PY{p}{)}
          \PY{n}{show}\PY{p}{(}\PY{p}{)}
\end{Verbatim}


    \begin{center}
    \adjustimage{max size={0.9\linewidth}{0.9\paperheight}}{output_99_0.png}
    \end{center}
    { \hspace*{\fill} \\}
    
    Nå kan vi kjøre på nytt og nytt, og set at noen ganger holder den seg
veldig nær 0, andre ganger går den et godt stykke ut til sidene. Det vi
gjør nå er å plott 1000 sånne kurver over hverandre, men vi gjør dem
også veldig gjennomsiktige. Der mange kurver ligger over hverandre er
det sannsynlig å havne og det blir en sterkere farge

    \begin{Verbatim}[commandchars=\\\{\}]
{\color{incolor}In [{\color{incolor}182}]:} \PY{n}{antall\PYZus{}kast} \PY{o}{=} \PY{l+m+mi}{1000}
          \PY{n}{antall\PYZus{}kurver} \PY{o}{=} \PY{l+m+mi}{1000}
          
          \PY{c+c1}{\PYZsh{} Ta vare på alle verdiene for alle kurvene}
          \PY{n}{historikk} \PY{o}{=} \PY{n}{np}\PY{o}{.}\PY{n}{zeros}\PY{p}{(}\PY{p}{(}\PY{n}{antall\PYZus{}kast}\PY{o}{+}\PY{l+m+mi}{1}\PY{p}{,} \PY{n}{antall\PYZus{}kurver}\PY{p}{)}\PY{p}{)}
          
          \PY{c+c1}{\PYZsh{} To løkker: En over kastene, og en over kurvene}
          \PY{k}{for} \PY{n}{kast} \PY{o+ow}{in} \PY{n+nb}{range}\PY{p}{(}\PY{n}{antall\PYZus{}kast}\PY{p}{)}\PY{p}{:}
              \PY{k}{for} \PY{n}{kurve} \PY{o+ow}{in} \PY{n+nb}{range}\PY{p}{(}\PY{n}{antall\PYZus{}kurver}\PY{p}{)}\PY{p}{:}
                  \PY{n}{historikk}\PY{p}{[}\PY{n}{kast}\PY{o}{+}\PY{l+m+mi}{1}\PY{p}{,} \PY{n}{kurve}\PY{p}{]} \PY{o}{=} \PY{n}{verdier}\PY{p}{[}\PY{n}{kast}\PY{p}{,} \PY{n}{kurve}\PY{p}{]} \PY{o}{+} \PY{l+m+mi}{2}\PY{o}{*}\PY{n}{randint}\PY{p}{(}\PY{l+m+mi}{2}\PY{p}{)} \PY{o}{\PYZhy{}} \PY{l+m+mi}{1}  \PY{c+c1}{\PYZsh{} Vi gjør dette for å få \PYZhy{}1 eller 1, istedenfor 0 eller 1}
          
          \PY{n}{plot}\PY{p}{(}\PY{n}{historikk}\PY{p}{,} \PY{n}{alpha}\PY{o}{=}\PY{l+m+mf}{0.01}\PY{p}{,} \PY{n}{color}\PY{o}{=}\PY{l+s+s1}{\PYZsq{}}\PY{l+s+s1}{red}\PY{l+s+s1}{\PYZsq{}}\PY{p}{)}
          
          \PY{c+c1}{\PYZsh{} Tegn på standardavvik}
          \PY{n}{x} \PY{o}{=} \PY{n}{arange}\PY{p}{(}\PY{n}{antall\PYZus{}kast}\PY{o}{+}\PY{l+m+mi}{1}\PY{p}{)}
          \PY{n}{plot}\PY{p}{(}\PY{n}{x}\PY{p}{,} \PY{n}{sqrt}\PY{p}{(}\PY{n}{x}\PY{p}{)}\PY{p}{,} \PY{n}{color}\PY{o}{=}\PY{l+s+s1}{\PYZsq{}}\PY{l+s+s1}{k}\PY{l+s+s1}{\PYZsq{}}\PY{p}{)}
          \PY{n}{plot}\PY{p}{(}\PY{n}{x}\PY{p}{,} \PY{o}{\PYZhy{}}\PY{n}{sqrt}\PY{p}{(}\PY{n}{x}\PY{p}{)}\PY{p}{,} \PY{n}{color}\PY{o}{=}\PY{l+s+s1}{\PYZsq{}}\PY{l+s+s1}{k}\PY{l+s+s1}{\PYZsq{}}\PY{p}{)}    
          \PY{n}{axis}\PY{p}{(}\PY{p}{(}\PY{l+m+mi}{0}\PY{p}{,} \PY{l+m+mi}{1000}\PY{p}{,} \PY{o}{\PYZhy{}}\PY{l+m+mi}{100}\PY{p}{,} \PY{l+m+mi}{100}\PY{p}{)}\PY{p}{)}
          \PY{n}{show}\PY{p}{(}\PY{p}{)}
\end{Verbatim}


    \begin{center}
    \adjustimage{max size={0.9\linewidth}{0.9\paperheight}}{output_101_0.png}
    \end{center}
    { \hspace*{\fill} \\}
    
    \paragraph{Eksempel:
Spørreundersøkelser}\label{eksempel-spuxf8rreundersuxf8kelser}

Når det er valgår blir det stadig publisert nye partibarometer som
prøver å si hva folk kommer til å stemme. Disse gjøres ved å spørre en
liten del av befolkning. Hvis vi velger tilfeldige mennesker til å delta
i undersøkelsen, så vil de vi spør mest sannsynlig være representative
for hele norges befolkning. Men hvor mange må vi spørre for å være
rimelig sikre på resultatet?

Dette er et eksempel på store talls lov, og akkurat som for eksempelet
vi så på med myntkast vil avviket fra den faktiske fordelingen bli
mindre jo fler vi spør. Det er alltid viktig å se på antall spurte i en
spørreundersøkelse for å skjønne hvor sikre vi kan være på at
resultatene faktisk er sanne, og ikke bare en tilfeldighet. På en
seriøst spørreundersøkelse bør det stå antall deltakere og feilmargin.
Feilmarginen er regnet ut ved hjelpe av matematikk og sier noe om hva
som er en rimelig feil å anta det finnes grunnet antall deltakere.

For eksempel står det på et partibarometer på NRK sine nettsider:

\begin{verbatim}
Partibarometer, februar 2018
Norstat for NRK. Periode 30/01–05/02. 943 intervjuer. Feilmarginer fra 0,9–3,4 pp.
\end{verbatim}

I denne undersøkelsen spurte de 943 hva de ville stemme om det var valg
imorgen. Basert på denne kommer man frem til at det er rimelig at
resultatene er 0.9-3.4 prosentpoeng feil. Altså vil vi ikke forvente at
noen av oppsluttningene er 5\% eller mer feil, for dette holder det med
1000 personer, men 3\% er fortsatt ganske mye når det kommer til
partioppslutning, på grunn av dette vil det ofte være ganske store
forskjeller i partibarometre, selv om de er tatt på akkurat samme
tidspunkt.

\textbf{Tankespørsmål:} Når vi kastet 1000 mynter kunne vi forvente en
feil i sannsynligheten på opptill ca 3\%. Hvorfor tror du denne \%-en er
så nær feilen NRK oppgir for sin spørreundersøkelse?

    \section{Prosjektoppgave: Monty Hall
Problemet}\label{prosjektoppgave-monty-hall-problemet}

Monty Hall problemet er en veldig kjent mattenøtt med et veldig
overraskende svar. Det er et problem som fører til mye engasjement og
diskusjon, og har selv en spennende historie der diskusjon rundt
matematikk ble satt på dagsorden i hele USA.

Monty Hall problemet er et flott problem å bruke i
Matematikkundervisningen, da det fører til god diskusjon rundt
sannsynlighet og tilfeldighet. Det er også fullt mulig å simulere
spillet i klasserommet for å teste om intuisjonen til elevene stemmer,
noe den ofte ikke gjør. Ved å simulere \emph{manuelt} kan man enkelt
gjennomføre spillet et par hundre ganger, men vi kan også simulere på
datamaskin for å få langt flere simuleringer. Dette gjør Monty Hall
problemet til en mulig prosjektoppgave, gruppearbeid eller kanskje et
tema for en fagdag.

Vi starter her med å presteree problemet og hvor det kommer fra.
Deretter går vi igjennom noen mulige forklaringer av løsningen før vi
til slutt dekker hvordan spillet kan simuleres både manuelt og ved hjelp
av programmering.

\subsubsection{Problemets historie}\label{problemets-historie}

Problemet ble formulert og løst allerede i 1975, men ble ikke allment
kjent på dette tidspunktet. Det var først når problemet ble formulert på
nytt og sendt til \emph{Ask Marilyn} i 1990 at ting virkelig tok av.
\emph{Ask Marilyn} er en kolonne i \emph{Parade magazine} der Marilyn
vos Savant besvarer diverse matematiske nøtter og logiske utfordringer.
Vos Savant er kjent for å ha blitt kåret til den høyeste målte IQen av
Guinnes rekordbok. (Marilyn vos Savant Kilde:
\href{https://en.wikipedia.org/wiki/Marilyn_vos_Savant\#/media/File:Marilyn_vos_Savant.jpg}{Wikimedia
Commons})

\subparagraph{Problemets formulering}\label{problemets-formulering}

Se for deg at du er på et gameshow, og du får velge mellom tre dører:
Bak én av dem er en bil; bak de andre, geiter. Du velger en dør, for
eksempel nummer 1. Verten, som vet hva som er bak dørene, åpner en av de
andre dørene, for eksempel nummer 3, som har en geit. Han sier nå til
deg, "Har du lyst å bytte til dør 2?". Er det til din fordel å bytte
dør?

(Dette er den originale formuleringen av problemet publisert i
\emph{Parade} i 1990, oversatt til norsk.)

Problemet er kjent som \emph{Monty Hall} problemet da Monty Hall var en
kjent gameshowvert på denne tiden.

\subparagraph{Debatten}\label{debatten}

Marilyn besvarte spørsmålet helt korrekt i sin kolonne. Men kort tid
etter at problemet og svarer ble publisert fikk \emph{Parade} en enorm
mengde leserbrev i protest og både lekfolk og fagfolk var skråsikre på
at hun tok feil - noen gikk så langt som å beskylde henne for å drive
med vranglære og for å aktivt ødelegge for matematikkundervisning i USA.
Dette førte til at Marilyn måtte returnere til det samme spørsmålet ved
flere anledninger og komme med mer detaljerte forklaringer. På
\href{http://marilynvossavant.com/game-show-problem/}{Marilyn's egne
nettside} kan dere lese de originale svarene Marilyn skrev, samt noen av
brevene \emph{Parade} mottok i protest og støtte.

\subparagraph{Trumfkortet}\label{trumfkortet}

Debatten rundt spørsmålet fortsatte helt til Marilyn til slutt spurte om
hjelp fra skoleklasser over hele USA til å rett og slett \emph{gjøre}
simuleringer for å komme frem til svaret. Kort tid etter var det gjort
nok forsøk og empirisk innsamling til å bekrefte at Marilyns svar var
helt rett hele tiden.

    \subsubsection{Å simulere én runde}\label{uxe5-simulere-uxe9n-runde}

Vi skal nå gjenta eksperimentet til Marilyn. Først gjør vi dette ved å
simulere Monty Hall problemet for hånd. Etter vi har gjort dette går vi
over til å gjøre det ved hjelp av programmering. Målet er å gjennomføre
en million eller flere eksperimenter, så vi kan være sikre på at vi har
funnet riktig svar.

    \paragraph{Simulere Monty Hall i klasserommet med
terning}\label{simulere-monty-hall-i-klasserommet-med-terning}

Vi kan simulere Monty Hall problemet med 3 pappkopper og et lite objekt
som kan legges under en kopp, for eksempel en liten ball, en
sammenrullet papirbit, en liten stein, eller hva som helst annet.

To personer samarbeider om simuleringen, den ene er verten og den andre
deltakerene. Først skjuler verten objektet under en av de tre koppene
uten at deltakeren ser det. Deretter velger deltakeren en av de tre
koppene. Verten løfter nå koppen og viser at objektet ikke er der.
Deltakeren kan nå velge om de vil bytte eller ikke.

Når vi skal simulere Monty Hall problemet mange ganger for å finne
sannsynligheter, er det viktig at vi er objektive. For eksempel må vi
passe på at vi ikke faller inn i mønstre fordi vi begynner å kjede oss
og slikt. Vi skal derfor bruke terninger.

Kall de tre koppene 'A', 'B' og 'C'. Først ruller verten en terning, om
det blir 1 eller 2 er premien under A, om det blir 3 eller 4 er det
under B og om det blir 5 eller 6 legges premien under C. Deretter ruller
deltakeren for hvilken han eller hun gjetter på etter samme system.
Etter gjettet viser verten en kopp uten en premie under. Spill 10 ganger
der man velger å bli, og 10 ganger der man velger å bytte. Hva er
resultatet?

    \paragraph{Simulere Monty Hall med
programmering}\label{simulere-monty-hall-med-programmering}

Vi skal nå lage et program som kan simulere Monty Hall problemet. Vi
starter med å lage et program som spiller én runde. Vi må være sikre på
at denne versjonen fungerer som den skal før vi går videre til å
simulere en lang rekke spill etterhverandre. Vi skriver ut informasjon
for hvert steg i prosessen for å se at alt går riktig for seg.

    \begin{Verbatim}[commandchars=\\\{\}]
{\color{incolor}In [{\color{incolor}183}]:} \PY{c+c1}{\PYZsh{} Hvilken strategi bruker vi, bytte eller bli?}
          \PY{n}{strategi} \PY{o}{=} \PY{l+s+s2}{\PYZdq{}}\PY{l+s+s2}{bytte}\PY{l+s+s2}{\PYZdq{}}
          
          \PY{c+c1}{\PYZsh{} Hvilken dør er premien bak?}
          \PY{n}{dører} \PY{o}{=} \PY{p}{[}\PY{l+s+s2}{\PYZdq{}}\PY{l+s+s2}{A}\PY{l+s+s2}{\PYZdq{}}\PY{p}{,} \PY{l+s+s2}{\PYZdq{}}\PY{l+s+s2}{B}\PY{l+s+s2}{\PYZdq{}}\PY{p}{,} \PY{l+s+s2}{\PYZdq{}}\PY{l+s+s2}{C}\PY{l+s+s2}{\PYZdq{}}\PY{p}{]}
          \PY{n}{fasit} \PY{o}{=} \PY{n}{choice}\PY{p}{(}\PY{n}{dører}\PY{p}{)}
          \PY{n+nb}{print}\PY{p}{(}\PY{l+s+s2}{\PYZdq{}}\PY{l+s+s2}{Premien er bak dør: }\PY{l+s+si}{\PYZob{}\PYZcb{}}\PY{l+s+s2}{\PYZdq{}}\PY{o}{.}\PY{n}{format}\PY{p}{(}\PY{n}{fasit}\PY{p}{)}\PY{p}{)}
          
          \PY{c+c1}{\PYZsh{} Plukk en dør tilfeldig}
          \PY{n}{førstevalg} \PY{o}{=} \PY{n}{choice}\PY{p}{(}\PY{n}{dører}\PY{p}{)}
          \PY{n+nb}{print}\PY{p}{(}\PY{l+s+s2}{\PYZdq{}}\PY{l+s+s2}{Dør }\PY{l+s+si}{\PYZob{}\PYZcb{}}\PY{l+s+s2}{ blir valgt.}\PY{l+s+s2}{\PYZdq{}}\PY{o}{.}\PY{n}{format}\PY{p}{(}\PY{n}{førstevalg}\PY{p}{)}\PY{p}{)}
          
          \PY{c+c1}{\PYZsh{} Verten viser frem en av dørene vi ikke har valgt, og viser at det er en geit}
          \PY{n}{shuffle}\PY{p}{(}\PY{n}{dører}\PY{p}{)}
          \PY{k}{for} \PY{n}{dør} \PY{o+ow}{in} \PY{n}{dører}\PY{p}{:}
              \PY{k}{if} \PY{n}{dør} \PY{o}{!=} \PY{n}{førstevalg}\PY{p}{:}
                  \PY{k}{if} \PY{n}{dør} \PY{o}{!=} \PY{n}{fasit}\PY{p}{:}
                      \PY{n}{geit\PYZus{}dør} \PY{o}{=} \PY{n}{dør}
                      \PY{n+nb}{print}\PY{p}{(}\PY{l+s+s2}{\PYZdq{}}\PY{l+s+s2}{Verten viser frem en geit bak dør}\PY{l+s+s2}{\PYZdq{}}\PY{p}{,} \PY{n}{geit\PYZus{}dør}\PY{p}{)}
                      \PY{k}{break}
          
          \PY{c+c1}{\PYZsh{} Vil vi bytte dør?}
          \PY{k}{if} \PY{n}{strategi} \PY{o}{==} \PY{l+s+s2}{\PYZdq{}}\PY{l+s+s2}{bytte}\PY{l+s+s2}{\PYZdq{}}\PY{p}{:}
              \PY{k}{for} \PY{n}{dør} \PY{o+ow}{in} \PY{n}{dører}\PY{p}{:}
                  \PY{k}{if} \PY{n}{dør} \PY{o}{!=} \PY{n}{førstevalg} \PY{o+ow}{and} \PY{n}{dør} \PY{o}{!=} \PY{n}{geit\PYZus{}dør}\PY{p}{:}
                      \PY{n}{endelig\PYZus{}valg} \PY{o}{=} \PY{n}{dør}
                      \PY{n+nb}{print}\PY{p}{(}\PY{l+s+s2}{\PYZdq{}}\PY{l+s+s2}{Du byttet til dør }\PY{l+s+si}{\PYZob{}\PYZcb{}}\PY{l+s+s2}{.}\PY{l+s+s2}{\PYZdq{}}\PY{o}{.}\PY{n}{format}\PY{p}{(}\PY{n}{endelig\PYZus{}valg}\PY{p}{)}\PY{p}{)}
          \PY{k}{else}\PY{p}{:}
              \PY{n}{endelig\PYZus{}valg} \PY{o}{=} \PY{n}{førstevalg}
              \PY{n+nb}{print}\PY{p}{(}\PY{l+s+s2}{\PYZdq{}}\PY{l+s+s2}{Du valgte å bli på dør }\PY{l+s+si}{\PYZob{}\PYZcb{}}\PY{l+s+s2}{\PYZdq{}}\PY{o}{.}\PY{n}{format}\PY{p}{(}\PY{n}{endelig\PYZus{}valg}\PY{p}{)}\PY{p}{)}
          
          \PY{c+c1}{\PYZsh{} Sjekk om vi vinner}
          \PY{k}{if} \PY{n}{endelig\PYZus{}valg} \PY{o}{==} \PY{n}{fasit}\PY{p}{:}
              \PY{n+nb}{print}\PY{p}{(}\PY{l+s+s2}{\PYZdq{}}\PY{l+s+s2}{Du vant bilen!}\PY{l+s+s2}{\PYZdq{}}\PY{p}{)}
          \PY{k}{else}\PY{p}{:}
              \PY{n+nb}{print}\PY{p}{(}\PY{l+s+s2}{\PYZdq{}}\PY{l+s+s2}{Du fikk en geit denne gangen}\PY{l+s+s2}{\PYZdq{}}\PY{p}{)}
\end{Verbatim}


    \begin{Verbatim}[commandchars=\\\{\}]
Premien er bak dør: A
Dør A blir valgt.
Verten viser frem en geit bak dør C
Du byttet til dør B.
Du fikk en geit denne gangen

    \end{Verbatim}

    \paragraph{Mange simuleringer}\label{mange-simuleringer}

Nå som vi har fått til å simulere en runde med Monty Hall problemet
gjenstår det kun å skrive om litt slik at vi kan gjennomføre andelen
spill, og finne sannsynligheten for å vinne med de ulike strategiene. Vi
lager en funksjon og bruker en løkke.

    \begin{Verbatim}[commandchars=\\\{\}]
{\color{incolor}In [{\color{incolor}184}]:} \PY{k}{def} \PY{n+nf}{montyhall}\PY{p}{(}\PY{n}{strategi}\PY{p}{)}\PY{p}{:}
              \PY{n}{dører} \PY{o}{=} \PY{p}{[}\PY{l+s+s2}{\PYZdq{}}\PY{l+s+s2}{A}\PY{l+s+s2}{\PYZdq{}}\PY{p}{,} \PY{l+s+s2}{\PYZdq{}}\PY{l+s+s2}{B}\PY{l+s+s2}{\PYZdq{}}\PY{p}{,} \PY{l+s+s2}{\PYZdq{}}\PY{l+s+s2}{C}\PY{l+s+s2}{\PYZdq{}}\PY{p}{]}
              \PY{n}{fasit} \PY{o}{=} \PY{n}{choice}\PY{p}{(}\PY{n}{dører}\PY{p}{)}
          
              \PY{n}{førstevalg} \PY{o}{=} \PY{n}{choice}\PY{p}{(}\PY{n}{dører}\PY{p}{)}
          
              \PY{n}{shuffle}\PY{p}{(}\PY{n}{dører}\PY{p}{)}
              \PY{k}{for} \PY{n}{dør} \PY{o+ow}{in} \PY{n}{dører}\PY{p}{:}
                  \PY{k}{if} \PY{n}{dør} \PY{o}{!=} \PY{n}{førstevalg}\PY{p}{:}
                      \PY{k}{if} \PY{n}{dør} \PY{o}{!=} \PY{n}{fasit}\PY{p}{:}
                          \PY{n}{geit\PYZus{}dør} \PY{o}{=} \PY{n}{dør}
                          \PY{k}{break}
          
              \PY{k}{if} \PY{n}{strategi} \PY{o}{==} \PY{l+s+s2}{\PYZdq{}}\PY{l+s+s2}{bytte}\PY{l+s+s2}{\PYZdq{}}\PY{p}{:}
                  \PY{k}{for} \PY{n}{dør} \PY{o+ow}{in} \PY{n}{dører}\PY{p}{:}
                      \PY{k}{if} \PY{n}{dør} \PY{o}{!=} \PY{n}{førstevalg} \PY{o+ow}{and} \PY{n}{dør} \PY{o}{!=} \PY{n}{geit\PYZus{}dør}\PY{p}{:}
                          \PY{n}{endelig\PYZus{}valg} \PY{o}{=} \PY{n}{dør}        
              \PY{k}{else}\PY{p}{:}
                  \PY{n}{endelig\PYZus{}valg} \PY{o}{=} \PY{n}{førstevalg}
              
              \PY{k}{if} \PY{n}{endelig\PYZus{}valg} \PY{o}{==} \PY{n}{fasit}\PY{p}{:}
                  \PY{k}{return} \PY{l+m+mi}{1}
              \PY{k}{else}\PY{p}{:}
                  \PY{k}{return} \PY{l+m+mi}{0}
\end{Verbatim}


    \begin{Verbatim}[commandchars=\\\{\}]
{\color{incolor}In [{\color{incolor}185}]:} \PY{n+nb}{print}\PY{p}{(}\PY{n}{montyhall}\PY{p}{(}\PY{l+s+s1}{\PYZsq{}}\PY{l+s+s1}{bytte}\PY{l+s+s1}{\PYZsq{}}\PY{p}{)}\PY{p}{)}
          \PY{n+nb}{print}\PY{p}{(}\PY{n}{montyhall}\PY{p}{(}\PY{l+s+s1}{\PYZsq{}}\PY{l+s+s1}{bytte}\PY{l+s+s1}{\PYZsq{}}\PY{p}{)}\PY{p}{)}
          \PY{n+nb}{print}\PY{p}{(}\PY{n}{montyhall}\PY{p}{(}\PY{l+s+s1}{\PYZsq{}}\PY{l+s+s1}{bytte}\PY{l+s+s1}{\PYZsq{}}\PY{p}{)}\PY{p}{)}
\end{Verbatim}


    \begin{Verbatim}[commandchars=\\\{\}]
1
1
1

    \end{Verbatim}

    \begin{Verbatim}[commandchars=\\\{\}]
{\color{incolor}In [{\color{incolor}187}]:} \PY{n}{strategi} \PY{o}{=} \PY{l+s+s1}{\PYZsq{}}\PY{l+s+s1}{bytte}\PY{l+s+s1}{\PYZsq{}}
          \PY{n}{antall\PYZus{}simuleringer} \PY{o}{=} \PY{l+m+mi}{1000000}
          \PY{n}{antall\PYZus{}seire} \PY{o}{=} \PY{l+m+mi}{0}
          
          \PY{k}{for} \PY{n}{runde} \PY{o+ow}{in} \PY{n+nb}{range}\PY{p}{(}\PY{n}{antall\PYZus{}simuleringer}\PY{p}{)}\PY{p}{:}
              \PY{n}{antall\PYZus{}seire} \PY{o}{+}\PY{o}{=} \PY{n}{montyhall}\PY{p}{(}\PY{n}{strategi}\PY{p}{)}
          
          \PY{c+c1}{\PYZsh{} For hver runde så enten vinner vi, eller så taper vi så vi vet at}
          \PY{n}{antall\PYZus{}tap} \PY{o}{=} \PY{n}{antall\PYZus{}simuleringer} \PY{o}{\PYZhy{}} \PY{n}{antall\PYZus{}seire}
              
          \PY{n+nb}{print}\PY{p}{(}\PY{l+s+s2}{\PYZdq{}}\PY{l+s+s2}{Du spiller med strategien: }\PY{l+s+si}{\PYZob{}\PYZcb{}}\PY{l+s+s2}{\PYZdq{}}\PY{o}{.}\PY{n}{format}\PY{p}{(}\PY{n}{strategi}\PY{p}{)}\PY{p}{)}
          \PY{n+nb}{print}\PY{p}{(}\PY{l+s+s2}{\PYZdq{}}\PY{l+s+s2}{Du spilte }\PY{l+s+si}{\PYZob{}\PYZcb{}}\PY{l+s+s2}{ antall runder.}\PY{l+s+s2}{\PYZdq{}}\PY{o}{.}\PY{n}{format}\PY{p}{(}\PY{n}{antall\PYZus{}simuleringer}\PY{p}{)}\PY{p}{)}
          \PY{n+nb}{print}\PY{p}{(}\PY{l+s+s2}{\PYZdq{}}\PY{l+s+s2}{Du vant }\PY{l+s+si}{\PYZob{}\PYZcb{}}\PY{l+s+s2}{ runder (}\PY{l+s+si}{\PYZob{}:.1\PYZpc{}\PYZcb{}}\PY{l+s+s2}{)}\PY{l+s+s2}{\PYZdq{}}\PY{o}{.}\PY{n}{format}\PY{p}{(}\PY{n}{antall\PYZus{}seire}\PY{p}{,} \PY{n}{antall\PYZus{}seire}\PY{o}{/}\PY{n}{antall\PYZus{}simuleringer}\PY{p}{)}\PY{p}{)}
          \PY{n+nb}{print}\PY{p}{(}\PY{l+s+s2}{\PYZdq{}}\PY{l+s+s2}{Du tapte }\PY{l+s+si}{\PYZob{}\PYZcb{}}\PY{l+s+s2}{ runder (}\PY{l+s+si}{\PYZob{}:.1\PYZpc{}\PYZcb{}}\PY{l+s+s2}{)}\PY{l+s+s2}{\PYZdq{}}\PY{o}{.}\PY{n}{format}\PY{p}{(}\PY{n}{antall\PYZus{}tap}\PY{p}{,} \PY{n}{antall\PYZus{}tap}\PY{o}{/}\PY{n}{antall\PYZus{}simuleringer}\PY{p}{)}\PY{p}{)}
\end{Verbatim}


    \begin{Verbatim}[commandchars=\\\{\}]
Du spiller med strategien: bytte
Du spilte 1000000 antall runder.
Du vant 665985 runder (66.6\%)
Du tapte 334015 runder (33.4\%)

    \end{Verbatim}


    % Add a bibliography block to the postdoc
    
    
    
    \end{document}
